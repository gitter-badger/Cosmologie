% !TEX TS-program = pdflatex
% !TEX encoding = UTF-8 Unicode

% This is a simple template for a LaTeX document using the "article" class.
% See "book", "report", "letter" for other types of document.

\documentclass[11pt]{article} % use larger type; default would be 10pt

\usepackage[utf8]{inputenc} % set input encoding (not needed with XeLaTeX)

%%% Examples of Article customizations
% These packages are optional, depending whether you want the features they provide.
% See the LaTeX Companion or other references for full information.

%%% PAGE DIMENSIONS
\usepackage{geometry} % to change the page dimensions
\geometry{a4paper} % or letterpaper (US) or a5paper or....
% \geometry{margin=2in} % for example, change the margins to 2 inches all round
% \geometry{landscape} % set up the page for landscape
%   read geometry.pdf for detailed page layout information

\usepackage{import}
\subimport{../common/}{header}

% \usepackage[parfill]{parskip} % Activate to begin paragraphs with an empty line rather than an indent

%%% PACKAGES
\usepackage{booktabs} % for much better looking tables
\usepackage{array} % for better arrays (eg matrices) in maths
\usepackage{paralist} % very flexible & customisable lists (eg. enumerate/itemize, etc.)
\usepackage{verbatim} % adds environment for commenting out blocks of text & for better verbatim
\usepackage{subfig} % make it possible to include more than one captioned figure/table in a single float
% These packages are all incorporated in the memoir class to one degree or another...

%%% HEADERS & FOOTERS
\usepackage{fancyhdr} % This should be set AFTER setting up the page geometry
\pagestyle{fancy} % options: empty , plain , fancy
\renewcommand{\headrulewidth}{0pt} % customise the layout...
\lhead{}\chead{}\rhead{}
\lfoot{}\cfoot{\thepage}\rfoot{}

%%% SECTION TITLE APPEARANCE
\usepackage{sectsty}
\allsectionsfont{\sffamily\mdseries\upshape} % (See the fntguide.pdf for font help)
% (This matches ConTeXt defaults)

%%% ToC (table of contents) APPEARANCE
\usepackage[nottoc,notlof,notlot]{tocbibind} % Put the bibliography in the ToC
\usepackage[titles,subfigure]{tocloft} % Alter the style of the Table of Contents
\renewcommand{\cftsecfont}{\rmfamily\mdseries\upshape}
\renewcommand{\cftsecpagefont}{\rmfamily\mdseries\upshape} % No bold!

%%% END Article customizations

%%% The "real" document content comes below...

\title{Notes}
\author{}
\date{} % Activate to display a given date or no date (if empty),
         % otherwise the current date is printed 

\begin{document}
\maketitle

\section{Temps propagation lumière dans un espace en expansion}

A envoie un signal à B situé à une distance $d$ ($x(A) = 0$ et $x(B) = d$) dans un espace en expansion selon $t \mapsto L(t)$. 

Pour la lumière

\begin{equation}
c \dd t = \dfrac{L(t)}{L(0)} dx
\end{equation}

Donc \begin{equation}
d = cL_0 \dint_0^T \dfrac{dt}{L(t)}
\end{equation}


Exemple 1 : $L(t) = L_0 e^{\alpha t}$ :

 \begin{equation}
d = c \dint_0^T e^{-\alpha t} dt = \dfrac{c}{\alpha} \left ( 1 - e^{-\alpha T} \right )
\end{equation}

Donc $T = \dfrac{1}{\alpha} \ln { \dfrac{1}{1-\alpha d/c}}$ si $d \leq c/\alpha$.
De plus $T = d/c + O(\alpha d/c)$ pour $\alpha \to 0$

Exemple 2 : $L(t) = L_0 \left ( \dfrac{t}{t_0} \right )^\beta$ :

 \begin{equation}
d = c t_0 ^\beta \dint_0^T t^{-\beta} dt = \dfrac{ct_0^\beta} {1-\beta}  T^{1-\beta}  \textrm{ si } \beta < 1
\end{equation}

Donc $T = \left [ (1-\beta) \dfrac{d}{ct_0^\beta} \right ] ^ { \left [1/(1-\beta)\right ] }$

Si $d$ est mesuré à $t_0$ :

 \begin{equation}
d = c t_0 ^\beta \dint_0^T t^{-\beta} dt = \dfrac{ct_0^\beta} {1-\beta}  \left ( T^{1-\beta} - t_0^{1-\beta} \right ) 
\end{equation}

\section{Horizons}

\textbf{Horizon des évènements} : distance maximale $R_{ev}$ de laquelle peut provenir un signal lumineux en un temps fini.

Exemple 1 : $L(t) = L_0 e^{\alpha t} \Rightarrow R_{ev} = c/\alpha$

Exemple 2 : $L_0 \left ( \dfrac{t}{t_0} \right )^\beta, \beta < 1 \Rightarrow R_{ev} = +\infty$

\textbf{Horizon des particules} : distance physique maximale à un instant $t_1$ de laquelle peut provenir un évènement issu de $t_0$.

\begin{equation}
R_{part} = L(t_1) c \dint_{t_0}^{t_1} \dfrac{dt}{L(t)}
\end{equation}

\section{Relation luminosité-z}

On définit $z$ tel que $1+z(t) = \dfrac{L(t)}{L_0}$.
On peut montrer qu'une source émettant de la lumière à la longueur d'onde $\lambda_0$ à $t=0$ est perçue comme émettant à la longueur d'onde $\lambda(t)$ au temps $t$ selon la relation $\lambda (t) = \lambda_0 (1+z(t))$

Une source émet $N$ photons par seconde en $x = 0$. Au temps $t$ on capte ce signal en un point originellement ($t = 0$) distant de $d$ de la source sur une "surface" $dl$ du front d'onde l(t). On mesure $dN'(t)$ photons par seconde.

On doit avoir :
$dN'(t) = \dfrac{dl}{l(t)} \dfrac{f(t)}{f_0} N$


La longueur du front d'onde au temps $t$ est donnée par $l(t)$. Dans un espace plat + 2D :

\begin{equation}
l(t) = 2\pi d \dfrac{L(t)}{L_0} = 2\pi d (1+z(t)) 
\end{equation}

Dans un espace sphérique 2D :
\begin{equation}
l(t) = 2\pi L(t) \sin \dfrac{d}{L_0} = 2\pi L_0 (1+z(t)) \sin \dfrac{d}{L_0} 
\end{equation}

L'énergie des photons décroit en $1+z$ (red shift) 

Espace plat 2D : 
\begin{equation}
\bar{L} (z) = \dfrac{\bar{L}}{2\pi d(1+z)^2}
\end{equation}

Espace sphérique 2D : 
\begin{equation}
\bar{L} (z) = \dfrac{\bar{L}}{2\pi L_0(1+z)^2 \sin\frac{d}{L_0}}
\end{equation}

Exemple : $L(t) = L_0 \left (\dfrac{t}{t_0} \right )  ^\beta$
\begin{equation}
t = t_0 \left (1+z(t) \right)^{1/\beta}
\end{equation}

Donc
\begin{equation}
d(z) = \dfrac{ct_0} {1-\beta} \left [ \left (1+z(t) \right)^{(1-\beta)/\beta} - 1 \right ]
\end{equation}

Et en espace plat :
\begin{equation}
\bar{L} (z) = \dfrac{\bar{L}(1-\beta)}{2\pi ct_0 \left [ (1+z)^{(1+\beta)/\beta}- (1+z)^2 \right ] } = \dfrac{\bar{L}(1-\beta)}{2\pi ct_0 \left(1+z \right ) \left [ (1+z)^{1/\beta}- (1+z) \right ] }
\end{equation}

Et de plus 
\begin{equation}
H(t) = \dfrac{\dot{L}}{L}(t) = \dfrac{\beta}{t_0}
\end{equation}

Donc 


\begin{equation}
\bar{L} (z) = \dfrac{\bar{L}(1-\beta)H_0}{2\pi \beta c\left(1+z \right ) \left [ (1+z)^{1/\beta}- (1+z) \right ] }
\end{equation}


\begin{figure}[H]
\centering
  \caption{L(z) en espace 2D plat pour $\beta = 1/3$ et $\beta = 2/3$}
% GNUPLOT: LaTeX picture
\setlength{\unitlength}{0.240900pt}
\ifx\plotpoint\undefined\newsavebox{\plotpoint}\fi
\begin{picture}(1500,900)(0,0)
\sbox{\plotpoint}{\rule[-0.200pt]{0.400pt}{0.400pt}}%
\put(171.0,131.0){\rule[-0.200pt]{4.818pt}{0.400pt}}
\put(151,131){\makebox(0,0)[r]{ 0}}
\put(1419.0,131.0){\rule[-0.200pt]{4.818pt}{0.400pt}}
\put(171.0,204.0){\rule[-0.200pt]{4.818pt}{0.400pt}}
\put(151,204){\makebox(0,0)[r]{ 0.2}}
\put(1419.0,204.0){\rule[-0.200pt]{4.818pt}{0.400pt}}
\put(171.0,277.0){\rule[-0.200pt]{4.818pt}{0.400pt}}
\put(151,277){\makebox(0,0)[r]{ 0.4}}
\put(1419.0,277.0){\rule[-0.200pt]{4.818pt}{0.400pt}}
\put(171.0,349.0){\rule[-0.200pt]{4.818pt}{0.400pt}}
\put(151,349){\makebox(0,0)[r]{ 0.6}}
\put(1419.0,349.0){\rule[-0.200pt]{4.818pt}{0.400pt}}
\put(171.0,422.0){\rule[-0.200pt]{4.818pt}{0.400pt}}
\put(151,422){\makebox(0,0)[r]{ 0.8}}
\put(1419.0,422.0){\rule[-0.200pt]{4.818pt}{0.400pt}}
\put(171.0,495.0){\rule[-0.200pt]{4.818pt}{0.400pt}}
\put(151,495){\makebox(0,0)[r]{ 1}}
\put(1419.0,495.0){\rule[-0.200pt]{4.818pt}{0.400pt}}
\put(171.0,568.0){\rule[-0.200pt]{4.818pt}{0.400pt}}
\put(151,568){\makebox(0,0)[r]{ 1.2}}
\put(1419.0,568.0){\rule[-0.200pt]{4.818pt}{0.400pt}}
\put(171.0,641.0){\rule[-0.200pt]{4.818pt}{0.400pt}}
\put(151,641){\makebox(0,0)[r]{ 1.4}}
\put(1419.0,641.0){\rule[-0.200pt]{4.818pt}{0.400pt}}
\put(171.0,713.0){\rule[-0.200pt]{4.818pt}{0.400pt}}
\put(151,713){\makebox(0,0)[r]{ 1.6}}
\put(1419.0,713.0){\rule[-0.200pt]{4.818pt}{0.400pt}}
\put(171.0,786.0){\rule[-0.200pt]{4.818pt}{0.400pt}}
\put(151,786){\makebox(0,0)[r]{ 1.8}}
\put(1419.0,786.0){\rule[-0.200pt]{4.818pt}{0.400pt}}
\put(171.0,859.0){\rule[-0.200pt]{4.818pt}{0.400pt}}
\put(151,859){\makebox(0,0)[r]{ 2}}
\put(1419.0,859.0){\rule[-0.200pt]{4.818pt}{0.400pt}}
\put(238.0,131.0){\rule[-0.200pt]{0.400pt}{4.818pt}}
\put(238,90){\makebox(0,0){ 0.1}}
\put(238.0,839.0){\rule[-0.200pt]{0.400pt}{4.818pt}}
\put(371.0,131.0){\rule[-0.200pt]{0.400pt}{4.818pt}}
\put(371,90){\makebox(0,0){ 0.2}}
\put(371.0,839.0){\rule[-0.200pt]{0.400pt}{4.818pt}}
\put(505.0,131.0){\rule[-0.200pt]{0.400pt}{4.818pt}}
\put(505,90){\makebox(0,0){ 0.3}}
\put(505.0,839.0){\rule[-0.200pt]{0.400pt}{4.818pt}}
\put(638.0,131.0){\rule[-0.200pt]{0.400pt}{4.818pt}}
\put(638,90){\makebox(0,0){ 0.4}}
\put(638.0,839.0){\rule[-0.200pt]{0.400pt}{4.818pt}}
\put(772.0,131.0){\rule[-0.200pt]{0.400pt}{4.818pt}}
\put(772,90){\makebox(0,0){ 0.5}}
\put(772.0,839.0){\rule[-0.200pt]{0.400pt}{4.818pt}}
\put(905.0,131.0){\rule[-0.200pt]{0.400pt}{4.818pt}}
\put(905,90){\makebox(0,0){ 0.6}}
\put(905.0,839.0){\rule[-0.200pt]{0.400pt}{4.818pt}}
\put(1039.0,131.0){\rule[-0.200pt]{0.400pt}{4.818pt}}
\put(1039,90){\makebox(0,0){ 0.7}}
\put(1039.0,839.0){\rule[-0.200pt]{0.400pt}{4.818pt}}
\put(1172.0,131.0){\rule[-0.200pt]{0.400pt}{4.818pt}}
\put(1172,90){\makebox(0,0){ 0.8}}
\put(1172.0,839.0){\rule[-0.200pt]{0.400pt}{4.818pt}}
\put(1306.0,131.0){\rule[-0.200pt]{0.400pt}{4.818pt}}
\put(1306,90){\makebox(0,0){ 0.9}}
\put(1306.0,839.0){\rule[-0.200pt]{0.400pt}{4.818pt}}
\put(1439.0,131.0){\rule[-0.200pt]{0.400pt}{4.818pt}}
\put(1439,90){\makebox(0,0){ 1}}
\put(1439.0,839.0){\rule[-0.200pt]{0.400pt}{4.818pt}}
\put(171.0,131.0){\rule[-0.200pt]{0.400pt}{175.375pt}}
\put(171.0,131.0){\rule[-0.200pt]{305.461pt}{0.400pt}}
\put(1439.0,131.0){\rule[-0.200pt]{0.400pt}{175.375pt}}
\put(171.0,859.0){\rule[-0.200pt]{305.461pt}{0.400pt}}
\put(30,495){\makebox(0,0){L(z)}}
\put(805,29){\makebox(0,0){z}}
\put(1279,819){\makebox(0,0)[r]{$\beta = 1/3$}}
\put(1299.0,819.0){\rule[-0.200pt]{24.090pt}{0.400pt}}
\put(171,473){\usebox{\plotpoint}}
\multiput(171.61,464.84)(0.447,-2.918){3}{\rule{0.108pt}{1.967pt}}
\multiput(170.17,468.92)(3.000,-9.918){2}{\rule{0.400pt}{0.983pt}}
\put(174.17,446){\rule{0.400pt}{2.700pt}}
\multiput(173.17,453.40)(2.000,-7.396){2}{\rule{0.400pt}{1.350pt}}
\multiput(176.61,438.94)(0.447,-2.472){3}{\rule{0.108pt}{1.700pt}}
\multiput(175.17,442.47)(3.000,-8.472){2}{\rule{0.400pt}{0.850pt}}
\put(179.17,422){\rule{0.400pt}{2.500pt}}
\multiput(178.17,428.81)(2.000,-6.811){2}{\rule{0.400pt}{1.250pt}}
\multiput(181.61,416.05)(0.447,-2.025){3}{\rule{0.108pt}{1.433pt}}
\multiput(180.17,419.03)(3.000,-7.025){2}{\rule{0.400pt}{0.717pt}}
\put(184.17,402){\rule{0.400pt}{2.100pt}}
\multiput(183.17,407.64)(2.000,-5.641){2}{\rule{0.400pt}{1.050pt}}
\multiput(186.61,396.05)(0.447,-2.025){3}{\rule{0.108pt}{1.433pt}}
\multiput(185.17,399.03)(3.000,-7.025){2}{\rule{0.400pt}{0.717pt}}
\put(189.17,384){\rule{0.400pt}{1.700pt}}
\multiput(188.17,388.47)(2.000,-4.472){2}{\rule{0.400pt}{0.850pt}}
\multiput(191.61,378.60)(0.447,-1.802){3}{\rule{0.108pt}{1.300pt}}
\multiput(190.17,381.30)(3.000,-6.302){2}{\rule{0.400pt}{0.650pt}}
\put(194.17,368){\rule{0.400pt}{1.500pt}}
\multiput(193.17,371.89)(2.000,-3.887){2}{\rule{0.400pt}{0.750pt}}
\multiput(196.61,363.16)(0.447,-1.579){3}{\rule{0.108pt}{1.167pt}}
\multiput(195.17,365.58)(3.000,-5.579){2}{\rule{0.400pt}{0.583pt}}
\put(199.17,353){\rule{0.400pt}{1.500pt}}
\multiput(198.17,356.89)(2.000,-3.887){2}{\rule{0.400pt}{0.750pt}}
\multiput(201.61,349.26)(0.447,-1.132){3}{\rule{0.108pt}{0.900pt}}
\multiput(200.17,351.13)(3.000,-4.132){2}{\rule{0.400pt}{0.450pt}}
\multiput(204.61,342.71)(0.447,-1.355){3}{\rule{0.108pt}{1.033pt}}
\multiput(203.17,344.86)(3.000,-4.855){2}{\rule{0.400pt}{0.517pt}}
\put(207.17,334){\rule{0.400pt}{1.300pt}}
\multiput(206.17,337.30)(2.000,-3.302){2}{\rule{0.400pt}{0.650pt}}
\multiput(209.61,330.82)(0.447,-0.909){3}{\rule{0.108pt}{0.767pt}}
\multiput(208.17,332.41)(3.000,-3.409){2}{\rule{0.400pt}{0.383pt}}
\put(212.17,323){\rule{0.400pt}{1.300pt}}
\multiput(211.17,326.30)(2.000,-3.302){2}{\rule{0.400pt}{0.650pt}}
\multiput(214.61,319.82)(0.447,-0.909){3}{\rule{0.108pt}{0.767pt}}
\multiput(213.17,321.41)(3.000,-3.409){2}{\rule{0.400pt}{0.383pt}}
\put(217.17,313){\rule{0.400pt}{1.100pt}}
\multiput(216.17,315.72)(2.000,-2.717){2}{\rule{0.400pt}{0.550pt}}
\multiput(219.61,309.82)(0.447,-0.909){3}{\rule{0.108pt}{0.767pt}}
\multiput(218.17,311.41)(3.000,-3.409){2}{\rule{0.400pt}{0.383pt}}
\put(222.17,304){\rule{0.400pt}{0.900pt}}
\multiput(221.17,306.13)(2.000,-2.132){2}{\rule{0.400pt}{0.450pt}}
\multiput(224.61,301.37)(0.447,-0.685){3}{\rule{0.108pt}{0.633pt}}
\multiput(223.17,302.69)(3.000,-2.685){2}{\rule{0.400pt}{0.317pt}}
\put(227.17,295){\rule{0.400pt}{1.100pt}}
\multiput(226.17,297.72)(2.000,-2.717){2}{\rule{0.400pt}{0.550pt}}
\multiput(229.61,292.37)(0.447,-0.685){3}{\rule{0.108pt}{0.633pt}}
\multiput(228.17,293.69)(3.000,-2.685){2}{\rule{0.400pt}{0.317pt}}
\multiput(232.00,289.95)(0.462,-0.447){3}{\rule{0.500pt}{0.108pt}}
\multiput(232.00,290.17)(1.962,-3.000){2}{\rule{0.250pt}{0.400pt}}
\put(235.17,284){\rule{0.400pt}{0.900pt}}
\multiput(234.17,286.13)(2.000,-2.132){2}{\rule{0.400pt}{0.450pt}}
\multiput(237.61,281.37)(0.447,-0.685){3}{\rule{0.108pt}{0.633pt}}
\multiput(236.17,282.69)(3.000,-2.685){2}{\rule{0.400pt}{0.317pt}}
\put(240.17,277){\rule{0.400pt}{0.700pt}}
\multiput(239.17,278.55)(2.000,-1.547){2}{\rule{0.400pt}{0.350pt}}
\multiput(242.00,275.95)(0.462,-0.447){3}{\rule{0.500pt}{0.108pt}}
\multiput(242.00,276.17)(1.962,-3.000){2}{\rule{0.250pt}{0.400pt}}
\put(245.17,271){\rule{0.400pt}{0.700pt}}
\multiput(244.17,272.55)(2.000,-1.547){2}{\rule{0.400pt}{0.350pt}}
\multiput(247.00,269.95)(0.462,-0.447){3}{\rule{0.500pt}{0.108pt}}
\multiput(247.00,270.17)(1.962,-3.000){2}{\rule{0.250pt}{0.400pt}}
\put(250.17,265){\rule{0.400pt}{0.700pt}}
\multiput(249.17,266.55)(2.000,-1.547){2}{\rule{0.400pt}{0.350pt}}
\multiput(252.00,263.95)(0.462,-0.447){3}{\rule{0.500pt}{0.108pt}}
\multiput(252.00,264.17)(1.962,-3.000){2}{\rule{0.250pt}{0.400pt}}
\put(255.17,259){\rule{0.400pt}{0.700pt}}
\multiput(254.17,260.55)(2.000,-1.547){2}{\rule{0.400pt}{0.350pt}}
\multiput(257.00,257.95)(0.462,-0.447){3}{\rule{0.500pt}{0.108pt}}
\multiput(257.00,258.17)(1.962,-3.000){2}{\rule{0.250pt}{0.400pt}}
\put(260,254.17){\rule{0.482pt}{0.400pt}}
\multiput(260.00,255.17)(1.000,-2.000){2}{\rule{0.241pt}{0.400pt}}
\multiput(262.00,252.95)(0.462,-0.447){3}{\rule{0.500pt}{0.108pt}}
\multiput(262.00,253.17)(1.962,-3.000){2}{\rule{0.250pt}{0.400pt}}
\put(265,249.17){\rule{0.700pt}{0.400pt}}
\multiput(265.00,250.17)(1.547,-2.000){2}{\rule{0.350pt}{0.400pt}}
\put(268,247.17){\rule{0.482pt}{0.400pt}}
\multiput(268.00,248.17)(1.000,-2.000){2}{\rule{0.241pt}{0.400pt}}
\put(270,245.17){\rule{0.700pt}{0.400pt}}
\multiput(270.00,246.17)(1.547,-2.000){2}{\rule{0.350pt}{0.400pt}}
\put(273.17,242){\rule{0.400pt}{0.700pt}}
\multiput(272.17,243.55)(2.000,-1.547){2}{\rule{0.400pt}{0.350pt}}
\put(275,240.17){\rule{0.700pt}{0.400pt}}
\multiput(275.00,241.17)(1.547,-2.000){2}{\rule{0.350pt}{0.400pt}}
\put(278,238.17){\rule{0.482pt}{0.400pt}}
\multiput(278.00,239.17)(1.000,-2.000){2}{\rule{0.241pt}{0.400pt}}
\put(280,236.17){\rule{0.700pt}{0.400pt}}
\multiput(280.00,237.17)(1.547,-2.000){2}{\rule{0.350pt}{0.400pt}}
\put(283,234.17){\rule{0.482pt}{0.400pt}}
\multiput(283.00,235.17)(1.000,-2.000){2}{\rule{0.241pt}{0.400pt}}
\put(285,232.17){\rule{0.700pt}{0.400pt}}
\multiput(285.00,233.17)(1.547,-2.000){2}{\rule{0.350pt}{0.400pt}}
\put(288,230.67){\rule{0.482pt}{0.400pt}}
\multiput(288.00,231.17)(1.000,-1.000){2}{\rule{0.241pt}{0.400pt}}
\put(290,229.17){\rule{0.700pt}{0.400pt}}
\multiput(290.00,230.17)(1.547,-2.000){2}{\rule{0.350pt}{0.400pt}}
\put(293,227.17){\rule{0.700pt}{0.400pt}}
\multiput(293.00,228.17)(1.547,-2.000){2}{\rule{0.350pt}{0.400pt}}
\put(296,225.67){\rule{0.482pt}{0.400pt}}
\multiput(296.00,226.17)(1.000,-1.000){2}{\rule{0.241pt}{0.400pt}}
\put(298,224.17){\rule{0.700pt}{0.400pt}}
\multiput(298.00,225.17)(1.547,-2.000){2}{\rule{0.350pt}{0.400pt}}
\put(301,222.17){\rule{0.482pt}{0.400pt}}
\multiput(301.00,223.17)(1.000,-2.000){2}{\rule{0.241pt}{0.400pt}}
\put(303,220.67){\rule{0.723pt}{0.400pt}}
\multiput(303.00,221.17)(1.500,-1.000){2}{\rule{0.361pt}{0.400pt}}
\put(306,219.17){\rule{0.482pt}{0.400pt}}
\multiput(306.00,220.17)(1.000,-2.000){2}{\rule{0.241pt}{0.400pt}}
\put(308,217.67){\rule{0.723pt}{0.400pt}}
\multiput(308.00,218.17)(1.500,-1.000){2}{\rule{0.361pt}{0.400pt}}
\put(311,216.17){\rule{0.482pt}{0.400pt}}
\multiput(311.00,217.17)(1.000,-2.000){2}{\rule{0.241pt}{0.400pt}}
\put(313,214.67){\rule{0.723pt}{0.400pt}}
\multiput(313.00,215.17)(1.500,-1.000){2}{\rule{0.361pt}{0.400pt}}
\put(316,213.67){\rule{0.482pt}{0.400pt}}
\multiput(316.00,214.17)(1.000,-1.000){2}{\rule{0.241pt}{0.400pt}}
\put(318,212.17){\rule{0.700pt}{0.400pt}}
\multiput(318.00,213.17)(1.547,-2.000){2}{\rule{0.350pt}{0.400pt}}
\put(321,210.67){\rule{0.482pt}{0.400pt}}
\multiput(321.00,211.17)(1.000,-1.000){2}{\rule{0.241pt}{0.400pt}}
\put(323,209.67){\rule{0.723pt}{0.400pt}}
\multiput(323.00,210.17)(1.500,-1.000){2}{\rule{0.361pt}{0.400pt}}
\put(326,208.67){\rule{0.723pt}{0.400pt}}
\multiput(326.00,209.17)(1.500,-1.000){2}{\rule{0.361pt}{0.400pt}}
\put(329,207.67){\rule{0.482pt}{0.400pt}}
\multiput(329.00,208.17)(1.000,-1.000){2}{\rule{0.241pt}{0.400pt}}
\put(331,206.17){\rule{0.700pt}{0.400pt}}
\multiput(331.00,207.17)(1.547,-2.000){2}{\rule{0.350pt}{0.400pt}}
\put(334,204.67){\rule{0.482pt}{0.400pt}}
\multiput(334.00,205.17)(1.000,-1.000){2}{\rule{0.241pt}{0.400pt}}
\put(336,203.67){\rule{0.723pt}{0.400pt}}
\multiput(336.00,204.17)(1.500,-1.000){2}{\rule{0.361pt}{0.400pt}}
\put(339,202.67){\rule{0.482pt}{0.400pt}}
\multiput(339.00,203.17)(1.000,-1.000){2}{\rule{0.241pt}{0.400pt}}
\put(341,201.67){\rule{0.723pt}{0.400pt}}
\multiput(341.00,202.17)(1.500,-1.000){2}{\rule{0.361pt}{0.400pt}}
\put(344,200.67){\rule{0.482pt}{0.400pt}}
\multiput(344.00,201.17)(1.000,-1.000){2}{\rule{0.241pt}{0.400pt}}
\put(346,199.67){\rule{0.723pt}{0.400pt}}
\multiput(346.00,200.17)(1.500,-1.000){2}{\rule{0.361pt}{0.400pt}}
\put(349,198.67){\rule{0.482pt}{0.400pt}}
\multiput(349.00,199.17)(1.000,-1.000){2}{\rule{0.241pt}{0.400pt}}
\put(351,197.67){\rule{0.723pt}{0.400pt}}
\multiput(351.00,198.17)(1.500,-1.000){2}{\rule{0.361pt}{0.400pt}}
\put(354,196.67){\rule{0.482pt}{0.400pt}}
\multiput(354.00,197.17)(1.000,-1.000){2}{\rule{0.241pt}{0.400pt}}
\put(356,195.67){\rule{0.723pt}{0.400pt}}
\multiput(356.00,196.17)(1.500,-1.000){2}{\rule{0.361pt}{0.400pt}}
\put(359,194.67){\rule{0.723pt}{0.400pt}}
\multiput(359.00,195.17)(1.500,-1.000){2}{\rule{0.361pt}{0.400pt}}
\put(362,193.67){\rule{0.482pt}{0.400pt}}
\multiput(362.00,194.17)(1.000,-1.000){2}{\rule{0.241pt}{0.400pt}}
\put(364,192.67){\rule{0.723pt}{0.400pt}}
\multiput(364.00,193.17)(1.500,-1.000){2}{\rule{0.361pt}{0.400pt}}
\put(369,191.67){\rule{0.723pt}{0.400pt}}
\multiput(369.00,192.17)(1.500,-1.000){2}{\rule{0.361pt}{0.400pt}}
\put(372,190.67){\rule{0.482pt}{0.400pt}}
\multiput(372.00,191.17)(1.000,-1.000){2}{\rule{0.241pt}{0.400pt}}
\put(374,189.67){\rule{0.723pt}{0.400pt}}
\multiput(374.00,190.17)(1.500,-1.000){2}{\rule{0.361pt}{0.400pt}}
\put(377,188.67){\rule{0.482pt}{0.400pt}}
\multiput(377.00,189.17)(1.000,-1.000){2}{\rule{0.241pt}{0.400pt}}
\put(367.0,193.0){\rule[-0.200pt]{0.482pt}{0.400pt}}
\put(382,187.67){\rule{0.482pt}{0.400pt}}
\multiput(382.00,188.17)(1.000,-1.000){2}{\rule{0.241pt}{0.400pt}}
\put(384,186.67){\rule{0.723pt}{0.400pt}}
\multiput(384.00,187.17)(1.500,-1.000){2}{\rule{0.361pt}{0.400pt}}
\put(387,185.67){\rule{0.723pt}{0.400pt}}
\multiput(387.00,186.17)(1.500,-1.000){2}{\rule{0.361pt}{0.400pt}}
\put(379.0,189.0){\rule[-0.200pt]{0.723pt}{0.400pt}}
\put(392,184.67){\rule{0.723pt}{0.400pt}}
\multiput(392.00,185.17)(1.500,-1.000){2}{\rule{0.361pt}{0.400pt}}
\put(395,183.67){\rule{0.482pt}{0.400pt}}
\multiput(395.00,184.17)(1.000,-1.000){2}{\rule{0.241pt}{0.400pt}}
\put(390.0,186.0){\rule[-0.200pt]{0.482pt}{0.400pt}}
\put(400,182.67){\rule{0.482pt}{0.400pt}}
\multiput(400.00,183.17)(1.000,-1.000){2}{\rule{0.241pt}{0.400pt}}
\put(402,181.67){\rule{0.723pt}{0.400pt}}
\multiput(402.00,182.17)(1.500,-1.000){2}{\rule{0.361pt}{0.400pt}}
\put(397.0,184.0){\rule[-0.200pt]{0.723pt}{0.400pt}}
\put(407,180.67){\rule{0.723pt}{0.400pt}}
\multiput(407.00,181.17)(1.500,-1.000){2}{\rule{0.361pt}{0.400pt}}
\put(410,179.67){\rule{0.482pt}{0.400pt}}
\multiput(410.00,180.17)(1.000,-1.000){2}{\rule{0.241pt}{0.400pt}}
\put(405.0,182.0){\rule[-0.200pt]{0.482pt}{0.400pt}}
\put(415,178.67){\rule{0.482pt}{0.400pt}}
\multiput(415.00,179.17)(1.000,-1.000){2}{\rule{0.241pt}{0.400pt}}
\put(412.0,180.0){\rule[-0.200pt]{0.723pt}{0.400pt}}
\put(420,177.67){\rule{0.723pt}{0.400pt}}
\multiput(420.00,178.17)(1.500,-1.000){2}{\rule{0.361pt}{0.400pt}}
\put(417.0,179.0){\rule[-0.200pt]{0.723pt}{0.400pt}}
\put(425,176.67){\rule{0.723pt}{0.400pt}}
\multiput(425.00,177.17)(1.500,-1.000){2}{\rule{0.361pt}{0.400pt}}
\put(423.0,178.0){\rule[-0.200pt]{0.482pt}{0.400pt}}
\put(430,175.67){\rule{0.723pt}{0.400pt}}
\multiput(430.00,176.17)(1.500,-1.000){2}{\rule{0.361pt}{0.400pt}}
\put(433,174.67){\rule{0.482pt}{0.400pt}}
\multiput(433.00,175.17)(1.000,-1.000){2}{\rule{0.241pt}{0.400pt}}
\put(428.0,177.0){\rule[-0.200pt]{0.482pt}{0.400pt}}
\put(438,173.67){\rule{0.482pt}{0.400pt}}
\multiput(438.00,174.17)(1.000,-1.000){2}{\rule{0.241pt}{0.400pt}}
\put(435.0,175.0){\rule[-0.200pt]{0.723pt}{0.400pt}}
\put(443,172.67){\rule{0.482pt}{0.400pt}}
\multiput(443.00,173.17)(1.000,-1.000){2}{\rule{0.241pt}{0.400pt}}
\put(440.0,174.0){\rule[-0.200pt]{0.723pt}{0.400pt}}
\put(451,171.67){\rule{0.482pt}{0.400pt}}
\multiput(451.00,172.17)(1.000,-1.000){2}{\rule{0.241pt}{0.400pt}}
\put(445.0,173.0){\rule[-0.200pt]{1.445pt}{0.400pt}}
\put(456,170.67){\rule{0.482pt}{0.400pt}}
\multiput(456.00,171.17)(1.000,-1.000){2}{\rule{0.241pt}{0.400pt}}
\put(453.0,172.0){\rule[-0.200pt]{0.723pt}{0.400pt}}
\put(461,169.67){\rule{0.482pt}{0.400pt}}
\multiput(461.00,170.17)(1.000,-1.000){2}{\rule{0.241pt}{0.400pt}}
\put(458.0,171.0){\rule[-0.200pt]{0.723pt}{0.400pt}}
\put(466,168.67){\rule{0.482pt}{0.400pt}}
\multiput(466.00,169.17)(1.000,-1.000){2}{\rule{0.241pt}{0.400pt}}
\put(463.0,170.0){\rule[-0.200pt]{0.723pt}{0.400pt}}
\put(473,167.67){\rule{0.723pt}{0.400pt}}
\multiput(473.00,168.17)(1.500,-1.000){2}{\rule{0.361pt}{0.400pt}}
\put(468.0,169.0){\rule[-0.200pt]{1.204pt}{0.400pt}}
\put(478,166.67){\rule{0.723pt}{0.400pt}}
\multiput(478.00,167.17)(1.500,-1.000){2}{\rule{0.361pt}{0.400pt}}
\put(476.0,168.0){\rule[-0.200pt]{0.482pt}{0.400pt}}
\put(486,165.67){\rule{0.723pt}{0.400pt}}
\multiput(486.00,166.17)(1.500,-1.000){2}{\rule{0.361pt}{0.400pt}}
\put(481.0,167.0){\rule[-0.200pt]{1.204pt}{0.400pt}}
\put(494,164.67){\rule{0.482pt}{0.400pt}}
\multiput(494.00,165.17)(1.000,-1.000){2}{\rule{0.241pt}{0.400pt}}
\put(489.0,166.0){\rule[-0.200pt]{1.204pt}{0.400pt}}
\put(501,163.67){\rule{0.723pt}{0.400pt}}
\multiput(501.00,164.17)(1.500,-1.000){2}{\rule{0.361pt}{0.400pt}}
\put(496.0,165.0){\rule[-0.200pt]{1.204pt}{0.400pt}}
\put(509,162.67){\rule{0.723pt}{0.400pt}}
\multiput(509.00,163.17)(1.500,-1.000){2}{\rule{0.361pt}{0.400pt}}
\put(504.0,164.0){\rule[-0.200pt]{1.204pt}{0.400pt}}
\put(517,161.67){\rule{0.482pt}{0.400pt}}
\multiput(517.00,162.17)(1.000,-1.000){2}{\rule{0.241pt}{0.400pt}}
\put(512.0,163.0){\rule[-0.200pt]{1.204pt}{0.400pt}}
\put(524,160.67){\rule{0.723pt}{0.400pt}}
\multiput(524.00,161.17)(1.500,-1.000){2}{\rule{0.361pt}{0.400pt}}
\put(519.0,162.0){\rule[-0.200pt]{1.204pt}{0.400pt}}
\put(534,159.67){\rule{0.723pt}{0.400pt}}
\multiput(534.00,160.17)(1.500,-1.000){2}{\rule{0.361pt}{0.400pt}}
\put(527.0,161.0){\rule[-0.200pt]{1.686pt}{0.400pt}}
\put(542,158.67){\rule{0.723pt}{0.400pt}}
\multiput(542.00,159.17)(1.500,-1.000){2}{\rule{0.361pt}{0.400pt}}
\put(537.0,160.0){\rule[-0.200pt]{1.204pt}{0.400pt}}
\put(552,157.67){\rule{0.723pt}{0.400pt}}
\multiput(552.00,158.17)(1.500,-1.000){2}{\rule{0.361pt}{0.400pt}}
\put(545.0,159.0){\rule[-0.200pt]{1.686pt}{0.400pt}}
\put(562,156.67){\rule{0.723pt}{0.400pt}}
\multiput(562.00,157.17)(1.500,-1.000){2}{\rule{0.361pt}{0.400pt}}
\put(555.0,158.0){\rule[-0.200pt]{1.686pt}{0.400pt}}
\put(572,155.67){\rule{0.723pt}{0.400pt}}
\multiput(572.00,156.17)(1.500,-1.000){2}{\rule{0.361pt}{0.400pt}}
\put(565.0,157.0){\rule[-0.200pt]{1.686pt}{0.400pt}}
\put(585,154.67){\rule{0.723pt}{0.400pt}}
\multiput(585.00,155.17)(1.500,-1.000){2}{\rule{0.361pt}{0.400pt}}
\put(575.0,156.0){\rule[-0.200pt]{2.409pt}{0.400pt}}
\put(595,153.67){\rule{0.723pt}{0.400pt}}
\multiput(595.00,154.17)(1.500,-1.000){2}{\rule{0.361pt}{0.400pt}}
\put(588.0,155.0){\rule[-0.200pt]{1.686pt}{0.400pt}}
\put(608,152.67){\rule{0.723pt}{0.400pt}}
\multiput(608.00,153.17)(1.500,-1.000){2}{\rule{0.361pt}{0.400pt}}
\put(598.0,154.0){\rule[-0.200pt]{2.409pt}{0.400pt}}
\put(623,151.67){\rule{0.723pt}{0.400pt}}
\multiput(623.00,152.17)(1.500,-1.000){2}{\rule{0.361pt}{0.400pt}}
\put(611.0,153.0){\rule[-0.200pt]{2.891pt}{0.400pt}}
\put(636,150.67){\rule{0.723pt}{0.400pt}}
\multiput(636.00,151.17)(1.500,-1.000){2}{\rule{0.361pt}{0.400pt}}
\put(626.0,152.0){\rule[-0.200pt]{2.409pt}{0.400pt}}
\put(654,149.67){\rule{0.482pt}{0.400pt}}
\multiput(654.00,150.17)(1.000,-1.000){2}{\rule{0.241pt}{0.400pt}}
\put(639.0,151.0){\rule[-0.200pt]{3.613pt}{0.400pt}}
\put(669,148.67){\rule{0.723pt}{0.400pt}}
\multiput(669.00,149.17)(1.500,-1.000){2}{\rule{0.361pt}{0.400pt}}
\put(656.0,150.0){\rule[-0.200pt]{3.132pt}{0.400pt}}
\put(687,147.67){\rule{0.482pt}{0.400pt}}
\multiput(687.00,148.17)(1.000,-1.000){2}{\rule{0.241pt}{0.400pt}}
\put(672.0,149.0){\rule[-0.200pt]{3.613pt}{0.400pt}}
\put(707,146.67){\rule{0.723pt}{0.400pt}}
\multiput(707.00,147.17)(1.500,-1.000){2}{\rule{0.361pt}{0.400pt}}
\put(689.0,148.0){\rule[-0.200pt]{4.336pt}{0.400pt}}
\put(727,145.67){\rule{0.723pt}{0.400pt}}
\multiput(727.00,146.17)(1.500,-1.000){2}{\rule{0.361pt}{0.400pt}}
\put(710.0,147.0){\rule[-0.200pt]{4.095pt}{0.400pt}}
\put(750,144.67){\rule{0.723pt}{0.400pt}}
\multiput(750.00,145.17)(1.500,-1.000){2}{\rule{0.361pt}{0.400pt}}
\put(730.0,146.0){\rule[-0.200pt]{4.818pt}{0.400pt}}
\put(776,143.67){\rule{0.482pt}{0.400pt}}
\multiput(776.00,144.17)(1.000,-1.000){2}{\rule{0.241pt}{0.400pt}}
\put(753.0,145.0){\rule[-0.200pt]{5.541pt}{0.400pt}}
\put(804,142.67){\rule{0.482pt}{0.400pt}}
\multiput(804.00,143.17)(1.000,-1.000){2}{\rule{0.241pt}{0.400pt}}
\put(778.0,144.0){\rule[-0.200pt]{6.263pt}{0.400pt}}
\put(837,141.67){\rule{0.482pt}{0.400pt}}
\multiput(837.00,142.17)(1.000,-1.000){2}{\rule{0.241pt}{0.400pt}}
\put(806.0,143.0){\rule[-0.200pt]{7.468pt}{0.400pt}}
\put(870,140.67){\rule{0.482pt}{0.400pt}}
\multiput(870.00,141.17)(1.000,-1.000){2}{\rule{0.241pt}{0.400pt}}
\put(839.0,142.0){\rule[-0.200pt]{7.468pt}{0.400pt}}
\put(910,139.67){\rule{0.723pt}{0.400pt}}
\multiput(910.00,140.17)(1.500,-1.000){2}{\rule{0.361pt}{0.400pt}}
\put(872.0,141.0){\rule[-0.200pt]{9.154pt}{0.400pt}}
\put(956,138.67){\rule{0.723pt}{0.400pt}}
\multiput(956.00,139.17)(1.500,-1.000){2}{\rule{0.361pt}{0.400pt}}
\put(913.0,140.0){\rule[-0.200pt]{10.359pt}{0.400pt}}
\put(1012,137.67){\rule{0.723pt}{0.400pt}}
\multiput(1012.00,138.17)(1.500,-1.000){2}{\rule{0.361pt}{0.400pt}}
\put(959.0,139.0){\rule[-0.200pt]{12.768pt}{0.400pt}}
\put(1076,136.67){\rule{0.482pt}{0.400pt}}
\multiput(1076.00,137.17)(1.000,-1.000){2}{\rule{0.241pt}{0.400pt}}
\put(1015.0,138.0){\rule[-0.200pt]{14.695pt}{0.400pt}}
\put(1154,135.67){\rule{0.723pt}{0.400pt}}
\multiput(1154.00,136.17)(1.500,-1.000){2}{\rule{0.361pt}{0.400pt}}
\put(1078.0,137.0){\rule[-0.200pt]{18.308pt}{0.400pt}}
\put(1256,134.67){\rule{0.723pt}{0.400pt}}
\multiput(1256.00,135.17)(1.500,-1.000){2}{\rule{0.361pt}{0.400pt}}
\put(1157.0,136.0){\rule[-0.200pt]{23.849pt}{0.400pt}}
\put(1391,133.67){\rule{0.482pt}{0.400pt}}
\multiput(1391.00,134.17)(1.000,-1.000){2}{\rule{0.241pt}{0.400pt}}
\put(1259.0,135.0){\rule[-0.200pt]{31.799pt}{0.400pt}}
\put(1393.0,134.0){\rule[-0.200pt]{11.081pt}{0.400pt}}
\put(1279,778){\makebox(0,0)[r]{$\beta = 2/3$}}
\multiput(1299,778)(20.756,0.000){5}{\usebox{\plotpoint}}
\put(1399,778){\usebox{\plotpoint}}
\put(171,840){\usebox{\plotpoint}}
\multiput(171,840)(2.211,-20.637){2}{\usebox{\plotpoint}}
\put(175.02,798.69){\usebox{\plotpoint}}
\put(177.00,778.03){\usebox{\plotpoint}}
\put(179.40,757.42){\usebox{\plotpoint}}
\put(181.32,736.76){\usebox{\plotpoint}}
\put(184.18,716.20){\usebox{\plotpoint}}
\put(186.38,695.57){\usebox{\plotpoint}}
\put(189.44,675.04){\usebox{\plotpoint}}
\put(192.22,654.48){\usebox{\plotpoint}}
\put(195.38,633.98){\usebox{\plotpoint}}
\put(199.06,613.56){\usebox{\plotpoint}}
\put(202.59,593.13){\usebox{\plotpoint}}
\put(207.19,572.89){\usebox{\plotpoint}}
\put(211.36,552.57){\usebox{\plotpoint}}
\put(215.99,532.37){\usebox{\plotpoint}}
\put(221.02,512.26){\usebox{\plotpoint}}
\put(226.57,492.28){\usebox{\plotpoint}}
\put(232.92,472.55){\usebox{\plotpoint}}
\put(239.97,453.07){\usebox{\plotpoint}}
\put(246.83,433.51){\usebox{\plotpoint}}
\put(255.15,414.55){\usebox{\plotpoint}}
\put(263.90,395.83){\usebox{\plotpoint}}
\put(273.89,377.77){\usebox{\plotpoint}}
\put(284.48,360.05){\usebox{\plotpoint}}
\put(296.48,343.28){\usebox{\plotpoint}}
\put(309.03,326.97){\usebox{\plotpoint}}
\put(322.69,311.46){\usebox{\plotpoint}}
\put(338.04,297.64){\usebox{\plotpoint}}
\put(353.72,284.19){\usebox{\plotpoint}}
\put(370.30,272.14){\usebox{\plotpoint}}
\put(387.28,260.91){\usebox{\plotpoint}}
\put(405.19,250.90){\usebox{\plotpoint}}
\put(423.72,242.28){\usebox{\plotpoint}}
\put(442.61,234.13){\usebox{\plotpoint}}
\put(461.88,226.56){\usebox{\plotpoint}}
\put(481.36,219.88){\usebox{\plotpoint}}
\put(500.81,213.10){\usebox{\plotpoint}}
\put(520.65,208.45){\usebox{\plotpoint}}
\put(540.30,203.00){\usebox{\plotpoint}}
\put(560.30,198.85){\usebox{\plotpoint}}
\put(580.21,194.93){\usebox{\plotpoint}}
\put(600.31,190.90){\usebox{\plotpoint}}
\put(620.46,187.18){\usebox{\plotpoint}}
\put(640.53,184.23){\usebox{\plotpoint}}
\put(660.71,181.15){\usebox{\plotpoint}}
\put(681.03,179.00){\usebox{\plotpoint}}
\put(701.18,176.41){\usebox{\plotpoint}}
\put(721.51,174.00){\usebox{\plotpoint}}
\put(741.87,172.00){\usebox{\plotpoint}}
\put(762.30,170.00){\usebox{\plotpoint}}
\put(782.73,168.00){\usebox{\plotpoint}}
\put(803.20,166.27){\usebox{\plotpoint}}
\put(823.68,165.00){\usebox{\plotpoint}}
\put(843.97,163.02){\usebox{\plotpoint}}
\put(864.56,162.00){\usebox{\plotpoint}}
\put(885.08,161.00){\usebox{\plotpoint}}
\put(905.36,159.00){\usebox{\plotpoint}}
\put(925.88,158.00){\usebox{\plotpoint}}
\put(946.47,157.00){\usebox{\plotpoint}}
\put(966.99,156.00){\usebox{\plotpoint}}
\put(987.58,155.00){\usebox{\plotpoint}}
\put(1008.10,154.00){\usebox{\plotpoint}}
\put(1028.62,153.00){\usebox{\plotpoint}}
\put(1049.38,153.00){\usebox{\plotpoint}}
\put(1069.97,152.00){\usebox{\plotpoint}}
\put(1090.49,151.00){\usebox{\plotpoint}}
\put(1111.08,150.00){\usebox{\plotpoint}}
\put(1131.84,150.00){\usebox{\plotpoint}}
\put(1152.36,149.00){\usebox{\plotpoint}}
\put(1172.88,148.00){\usebox{\plotpoint}}
\put(1193.63,148.00){\usebox{\plotpoint}}
\put(1214.23,147.00){\usebox{\plotpoint}}
\put(1234.98,147.00){\usebox{\plotpoint}}
\put(1255.58,146.00){\usebox{\plotpoint}}
\put(1276.33,146.00){\usebox{\plotpoint}}
\put(1296.85,145.00){\usebox{\plotpoint}}
\put(1317.61,145.00){\usebox{\plotpoint}}
\put(1338.13,144.00){\usebox{\plotpoint}}
\put(1358.88,144.00){\usebox{\plotpoint}}
\put(1379.64,144.00){\usebox{\plotpoint}}
\put(1400.23,143.00){\usebox{\plotpoint}}
\put(1420.99,143.00){\usebox{\plotpoint}}
\put(1439,143){\usebox{\plotpoint}}
\put(171.0,131.0){\rule[-0.200pt]{0.400pt}{175.375pt}}
\put(171.0,131.0){\rule[-0.200pt]{305.461pt}{0.400pt}}
\put(1439.0,131.0){\rule[-0.200pt]{0.400pt}{175.375pt}}
\put(171.0,859.0){\rule[-0.200pt]{305.461pt}{0.400pt}}
\end{picture}

\end{figure}


En sphérique :

\begin{equation}
\bar{L} (z) = \dfrac{\bar{L}}{2\pi L_0 (1+z)^2 \sin\dfrac{c\beta} {H_0 L_0(1-\beta)} \left [ \left (1+z \right)^{(1-\beta)/\beta} - 1 \right ]}
\end{equation}

Exemple : $L(t) = L_0 e^{\alpha t}$
\begin{equation}
t = \dfrac{1}{\alpha} \ln {\left (1+z(t) \right )}
\end{equation}

Donc
\begin{equation}
d(z) = \dfrac{c}{\alpha} \dfrac{z(t)}{1+z(t)}
\end{equation}

Et si la courbure est nulle :

\begin{equation}
\bar{L} (z) = \dfrac{H_0 \bar{L}}{2\pi c z(1+z) }
\end{equation}

Pour une géométrie sphérique :

\begin{equation}
\bar{L} (z) = \dfrac{\bar{L}}{2\pi L_0(1+z)^2 \sin\frac{c}{H_0 L_0} \dfrac{z}{1+z}}
\end{equation}

\begin{figure}[H]
\centering
  \caption{L(z) en espace 2D plat pour $\beta = 1/3$ et $\beta = 2/3$}
% GNUPLOT: LaTeX picture
\setlength{\unitlength}{0.240900pt}
\ifx\plotpoint\undefined\newsavebox{\plotpoint}\fi
\begin{picture}(1500,900)(0,0)
\sbox{\plotpoint}{\rule[-0.200pt]{0.400pt}{0.400pt}}%
\put(211.0,131.0){\rule[-0.200pt]{4.818pt}{0.400pt}}
\put(191,131){\makebox(0,0)[r]{ 0.001}}
\put(1419.0,131.0){\rule[-0.200pt]{4.818pt}{0.400pt}}
\put(211.0,204.0){\rule[-0.200pt]{2.409pt}{0.400pt}}
\put(1429.0,204.0){\rule[-0.200pt]{2.409pt}{0.400pt}}
\put(211.0,247.0){\rule[-0.200pt]{2.409pt}{0.400pt}}
\put(1429.0,247.0){\rule[-0.200pt]{2.409pt}{0.400pt}}
\put(211.0,277.0){\rule[-0.200pt]{2.409pt}{0.400pt}}
\put(1429.0,277.0){\rule[-0.200pt]{2.409pt}{0.400pt}}
\put(211.0,301.0){\rule[-0.200pt]{2.409pt}{0.400pt}}
\put(1429.0,301.0){\rule[-0.200pt]{2.409pt}{0.400pt}}
\put(211.0,320.0){\rule[-0.200pt]{2.409pt}{0.400pt}}
\put(1429.0,320.0){\rule[-0.200pt]{2.409pt}{0.400pt}}
\put(211.0,336.0){\rule[-0.200pt]{2.409pt}{0.400pt}}
\put(1429.0,336.0){\rule[-0.200pt]{2.409pt}{0.400pt}}
\put(211.0,350.0){\rule[-0.200pt]{2.409pt}{0.400pt}}
\put(1429.0,350.0){\rule[-0.200pt]{2.409pt}{0.400pt}}
\put(211.0,363.0){\rule[-0.200pt]{2.409pt}{0.400pt}}
\put(1429.0,363.0){\rule[-0.200pt]{2.409pt}{0.400pt}}
\put(211.0,374.0){\rule[-0.200pt]{4.818pt}{0.400pt}}
\put(191,374){\makebox(0,0)[r]{ 0.01}}
\put(1419.0,374.0){\rule[-0.200pt]{4.818pt}{0.400pt}}
\put(211.0,447.0){\rule[-0.200pt]{2.409pt}{0.400pt}}
\put(1429.0,447.0){\rule[-0.200pt]{2.409pt}{0.400pt}}
\put(211.0,489.0){\rule[-0.200pt]{2.409pt}{0.400pt}}
\put(1429.0,489.0){\rule[-0.200pt]{2.409pt}{0.400pt}}
\put(211.0,520.0){\rule[-0.200pt]{2.409pt}{0.400pt}}
\put(1429.0,520.0){\rule[-0.200pt]{2.409pt}{0.400pt}}
\put(211.0,543.0){\rule[-0.200pt]{2.409pt}{0.400pt}}
\put(1429.0,543.0){\rule[-0.200pt]{2.409pt}{0.400pt}}
\put(211.0,562.0){\rule[-0.200pt]{2.409pt}{0.400pt}}
\put(1429.0,562.0){\rule[-0.200pt]{2.409pt}{0.400pt}}
\put(211.0,579.0){\rule[-0.200pt]{2.409pt}{0.400pt}}
\put(1429.0,579.0){\rule[-0.200pt]{2.409pt}{0.400pt}}
\put(211.0,593.0){\rule[-0.200pt]{2.409pt}{0.400pt}}
\put(1429.0,593.0){\rule[-0.200pt]{2.409pt}{0.400pt}}
\put(211.0,605.0){\rule[-0.200pt]{2.409pt}{0.400pt}}
\put(1429.0,605.0){\rule[-0.200pt]{2.409pt}{0.400pt}}
\put(211.0,616.0){\rule[-0.200pt]{4.818pt}{0.400pt}}
\put(191,616){\makebox(0,0)[r]{ 0.1}}
\put(1419.0,616.0){\rule[-0.200pt]{4.818pt}{0.400pt}}
\put(211.0,689.0){\rule[-0.200pt]{2.409pt}{0.400pt}}
\put(1429.0,689.0){\rule[-0.200pt]{2.409pt}{0.400pt}}
\put(211.0,732.0){\rule[-0.200pt]{2.409pt}{0.400pt}}
\put(1429.0,732.0){\rule[-0.200pt]{2.409pt}{0.400pt}}
\put(211.0,762.0){\rule[-0.200pt]{2.409pt}{0.400pt}}
\put(1429.0,762.0){\rule[-0.200pt]{2.409pt}{0.400pt}}
\put(211.0,786.0){\rule[-0.200pt]{2.409pt}{0.400pt}}
\put(1429.0,786.0){\rule[-0.200pt]{2.409pt}{0.400pt}}
\put(211.0,805.0){\rule[-0.200pt]{2.409pt}{0.400pt}}
\put(1429.0,805.0){\rule[-0.200pt]{2.409pt}{0.400pt}}
\put(211.0,821.0){\rule[-0.200pt]{2.409pt}{0.400pt}}
\put(1429.0,821.0){\rule[-0.200pt]{2.409pt}{0.400pt}}
\put(211.0,835.0){\rule[-0.200pt]{2.409pt}{0.400pt}}
\put(1429.0,835.0){\rule[-0.200pt]{2.409pt}{0.400pt}}
\put(211.0,848.0){\rule[-0.200pt]{2.409pt}{0.400pt}}
\put(1429.0,848.0){\rule[-0.200pt]{2.409pt}{0.400pt}}
\put(211.0,859.0){\rule[-0.200pt]{4.818pt}{0.400pt}}
\put(191,859){\makebox(0,0)[r]{ 1}}
\put(1419.0,859.0){\rule[-0.200pt]{4.818pt}{0.400pt}}
\put(211.0,131.0){\rule[-0.200pt]{0.400pt}{4.818pt}}
\put(211,90){\makebox(0,0){ 0.1}}
\put(211.0,839.0){\rule[-0.200pt]{0.400pt}{4.818pt}}
\put(495.0,131.0){\rule[-0.200pt]{0.400pt}{2.409pt}}
\put(495.0,849.0){\rule[-0.200pt]{0.400pt}{2.409pt}}
\put(661.0,131.0){\rule[-0.200pt]{0.400pt}{2.409pt}}
\put(661.0,849.0){\rule[-0.200pt]{0.400pt}{2.409pt}}
\put(779.0,131.0){\rule[-0.200pt]{0.400pt}{2.409pt}}
\put(779.0,849.0){\rule[-0.200pt]{0.400pt}{2.409pt}}
\put(871.0,131.0){\rule[-0.200pt]{0.400pt}{2.409pt}}
\put(871.0,849.0){\rule[-0.200pt]{0.400pt}{2.409pt}}
\put(945.0,131.0){\rule[-0.200pt]{0.400pt}{2.409pt}}
\put(945.0,849.0){\rule[-0.200pt]{0.400pt}{2.409pt}}
\put(1009.0,131.0){\rule[-0.200pt]{0.400pt}{2.409pt}}
\put(1009.0,849.0){\rule[-0.200pt]{0.400pt}{2.409pt}}
\put(1063.0,131.0){\rule[-0.200pt]{0.400pt}{2.409pt}}
\put(1063.0,849.0){\rule[-0.200pt]{0.400pt}{2.409pt}}
\put(1112.0,131.0){\rule[-0.200pt]{0.400pt}{2.409pt}}
\put(1112.0,849.0){\rule[-0.200pt]{0.400pt}{2.409pt}}
\put(1155.0,131.0){\rule[-0.200pt]{0.400pt}{4.818pt}}
\put(1155,90){\makebox(0,0){ 1}}
\put(1155.0,839.0){\rule[-0.200pt]{0.400pt}{4.818pt}}
\put(1439.0,131.0){\rule[-0.200pt]{0.400pt}{2.409pt}}
\put(1439.0,849.0){\rule[-0.200pt]{0.400pt}{2.409pt}}
\put(211.0,131.0){\rule[-0.200pt]{0.400pt}{175.375pt}}
\put(211.0,131.0){\rule[-0.200pt]{295.825pt}{0.400pt}}
\put(1439.0,131.0){\rule[-0.200pt]{0.400pt}{175.375pt}}
\put(211.0,859.0){\rule[-0.200pt]{295.825pt}{0.400pt}}
\put(30,495){\makebox(0,0){L(z)}}
\put(825,29){\makebox(0,0){z}}
\put(1279,819){\makebox(0,0)[r]{$\beta = 1/3$}}
\put(1299.0,819.0){\rule[-0.200pt]{24.090pt}{0.400pt}}
\put(211,767){\usebox{\plotpoint}}
\put(211,765.67){\rule{0.482pt}{0.400pt}}
\multiput(211.00,766.17)(1.000,-1.000){2}{\rule{0.241pt}{0.400pt}}
\put(213,764.67){\rule{0.723pt}{0.400pt}}
\multiput(213.00,765.17)(1.500,-1.000){2}{\rule{0.361pt}{0.400pt}}
\put(218,763.67){\rule{0.723pt}{0.400pt}}
\multiput(218.00,764.17)(1.500,-1.000){2}{\rule{0.361pt}{0.400pt}}
\put(221,762.67){\rule{0.482pt}{0.400pt}}
\multiput(221.00,763.17)(1.000,-1.000){2}{\rule{0.241pt}{0.400pt}}
\put(223,761.67){\rule{0.723pt}{0.400pt}}
\multiput(223.00,762.17)(1.500,-1.000){2}{\rule{0.361pt}{0.400pt}}
\put(226,760.67){\rule{0.482pt}{0.400pt}}
\multiput(226.00,761.17)(1.000,-1.000){2}{\rule{0.241pt}{0.400pt}}
\put(216.0,765.0){\rule[-0.200pt]{0.482pt}{0.400pt}}
\put(231,759.67){\rule{0.482pt}{0.400pt}}
\multiput(231.00,760.17)(1.000,-1.000){2}{\rule{0.241pt}{0.400pt}}
\put(233,758.67){\rule{0.723pt}{0.400pt}}
\multiput(233.00,759.17)(1.500,-1.000){2}{\rule{0.361pt}{0.400pt}}
\put(236,757.67){\rule{0.482pt}{0.400pt}}
\multiput(236.00,758.17)(1.000,-1.000){2}{\rule{0.241pt}{0.400pt}}
\put(228.0,761.0){\rule[-0.200pt]{0.723pt}{0.400pt}}
\put(241,756.67){\rule{0.482pt}{0.400pt}}
\multiput(241.00,757.17)(1.000,-1.000){2}{\rule{0.241pt}{0.400pt}}
\put(243,755.67){\rule{0.482pt}{0.400pt}}
\multiput(243.00,756.17)(1.000,-1.000){2}{\rule{0.241pt}{0.400pt}}
\put(245,754.67){\rule{0.723pt}{0.400pt}}
\multiput(245.00,755.17)(1.500,-1.000){2}{\rule{0.361pt}{0.400pt}}
\put(248,753.67){\rule{0.482pt}{0.400pt}}
\multiput(248.00,754.17)(1.000,-1.000){2}{\rule{0.241pt}{0.400pt}}
\put(238.0,758.0){\rule[-0.200pt]{0.723pt}{0.400pt}}
\put(253,752.67){\rule{0.482pt}{0.400pt}}
\multiput(253.00,753.17)(1.000,-1.000){2}{\rule{0.241pt}{0.400pt}}
\put(255,751.67){\rule{0.723pt}{0.400pt}}
\multiput(255.00,752.17)(1.500,-1.000){2}{\rule{0.361pt}{0.400pt}}
\put(258,750.67){\rule{0.482pt}{0.400pt}}
\multiput(258.00,751.17)(1.000,-1.000){2}{\rule{0.241pt}{0.400pt}}
\put(260,749.67){\rule{0.723pt}{0.400pt}}
\multiput(260.00,750.17)(1.500,-1.000){2}{\rule{0.361pt}{0.400pt}}
\put(250.0,754.0){\rule[-0.200pt]{0.723pt}{0.400pt}}
\put(265,748.67){\rule{0.723pt}{0.400pt}}
\multiput(265.00,749.17)(1.500,-1.000){2}{\rule{0.361pt}{0.400pt}}
\put(268,747.67){\rule{0.482pt}{0.400pt}}
\multiput(268.00,748.17)(1.000,-1.000){2}{\rule{0.241pt}{0.400pt}}
\put(270,746.67){\rule{0.723pt}{0.400pt}}
\multiput(270.00,747.17)(1.500,-1.000){2}{\rule{0.361pt}{0.400pt}}
\put(273,745.67){\rule{0.482pt}{0.400pt}}
\multiput(273.00,746.17)(1.000,-1.000){2}{\rule{0.241pt}{0.400pt}}
\put(263.0,750.0){\rule[-0.200pt]{0.482pt}{0.400pt}}
\put(277,744.67){\rule{0.723pt}{0.400pt}}
\multiput(277.00,745.17)(1.500,-1.000){2}{\rule{0.361pt}{0.400pt}}
\put(280,743.67){\rule{0.482pt}{0.400pt}}
\multiput(280.00,744.17)(1.000,-1.000){2}{\rule{0.241pt}{0.400pt}}
\put(282,742.67){\rule{0.723pt}{0.400pt}}
\multiput(282.00,743.17)(1.500,-1.000){2}{\rule{0.361pt}{0.400pt}}
\put(285,741.67){\rule{0.482pt}{0.400pt}}
\multiput(285.00,742.17)(1.000,-1.000){2}{\rule{0.241pt}{0.400pt}}
\put(275.0,746.0){\rule[-0.200pt]{0.482pt}{0.400pt}}
\put(290,740.67){\rule{0.482pt}{0.400pt}}
\multiput(290.00,741.17)(1.000,-1.000){2}{\rule{0.241pt}{0.400pt}}
\put(292,739.67){\rule{0.723pt}{0.400pt}}
\multiput(292.00,740.17)(1.500,-1.000){2}{\rule{0.361pt}{0.400pt}}
\put(295,738.67){\rule{0.482pt}{0.400pt}}
\multiput(295.00,739.17)(1.000,-1.000){2}{\rule{0.241pt}{0.400pt}}
\put(297,737.67){\rule{0.723pt}{0.400pt}}
\multiput(297.00,738.17)(1.500,-1.000){2}{\rule{0.361pt}{0.400pt}}
\put(287.0,742.0){\rule[-0.200pt]{0.723pt}{0.400pt}}
\put(302,736.67){\rule{0.723pt}{0.400pt}}
\multiput(302.00,737.17)(1.500,-1.000){2}{\rule{0.361pt}{0.400pt}}
\put(305,735.67){\rule{0.482pt}{0.400pt}}
\multiput(305.00,736.17)(1.000,-1.000){2}{\rule{0.241pt}{0.400pt}}
\put(307,734.67){\rule{0.482pt}{0.400pt}}
\multiput(307.00,735.17)(1.000,-1.000){2}{\rule{0.241pt}{0.400pt}}
\put(309,733.67){\rule{0.723pt}{0.400pt}}
\multiput(309.00,734.17)(1.500,-1.000){2}{\rule{0.361pt}{0.400pt}}
\put(300.0,738.0){\rule[-0.200pt]{0.482pt}{0.400pt}}
\put(314,732.67){\rule{0.723pt}{0.400pt}}
\multiput(314.00,733.17)(1.500,-1.000){2}{\rule{0.361pt}{0.400pt}}
\put(317,731.67){\rule{0.482pt}{0.400pt}}
\multiput(317.00,732.17)(1.000,-1.000){2}{\rule{0.241pt}{0.400pt}}
\put(319,730.67){\rule{0.723pt}{0.400pt}}
\multiput(319.00,731.17)(1.500,-1.000){2}{\rule{0.361pt}{0.400pt}}
\put(322,729.67){\rule{0.482pt}{0.400pt}}
\multiput(322.00,730.17)(1.000,-1.000){2}{\rule{0.241pt}{0.400pt}}
\put(324,728.67){\rule{0.723pt}{0.400pt}}
\multiput(324.00,729.17)(1.500,-1.000){2}{\rule{0.361pt}{0.400pt}}
\put(312.0,734.0){\rule[-0.200pt]{0.482pt}{0.400pt}}
\put(329,727.67){\rule{0.723pt}{0.400pt}}
\multiput(329.00,728.17)(1.500,-1.000){2}{\rule{0.361pt}{0.400pt}}
\put(332,726.67){\rule{0.482pt}{0.400pt}}
\multiput(332.00,727.17)(1.000,-1.000){2}{\rule{0.241pt}{0.400pt}}
\put(334,725.67){\rule{0.723pt}{0.400pt}}
\multiput(334.00,726.17)(1.500,-1.000){2}{\rule{0.361pt}{0.400pt}}
\put(337,724.67){\rule{0.482pt}{0.400pt}}
\multiput(337.00,725.17)(1.000,-1.000){2}{\rule{0.241pt}{0.400pt}}
\put(327.0,729.0){\rule[-0.200pt]{0.482pt}{0.400pt}}
\put(341,723.67){\rule{0.723pt}{0.400pt}}
\multiput(341.00,724.17)(1.500,-1.000){2}{\rule{0.361pt}{0.400pt}}
\put(344,722.67){\rule{0.482pt}{0.400pt}}
\multiput(344.00,723.17)(1.000,-1.000){2}{\rule{0.241pt}{0.400pt}}
\put(346,721.67){\rule{0.723pt}{0.400pt}}
\multiput(346.00,722.17)(1.500,-1.000){2}{\rule{0.361pt}{0.400pt}}
\put(349,720.67){\rule{0.482pt}{0.400pt}}
\multiput(349.00,721.17)(1.000,-1.000){2}{\rule{0.241pt}{0.400pt}}
\put(351,719.67){\rule{0.723pt}{0.400pt}}
\multiput(351.00,720.17)(1.500,-1.000){2}{\rule{0.361pt}{0.400pt}}
\put(339.0,725.0){\rule[-0.200pt]{0.482pt}{0.400pt}}
\put(356,718.67){\rule{0.723pt}{0.400pt}}
\multiput(356.00,719.17)(1.500,-1.000){2}{\rule{0.361pt}{0.400pt}}
\put(359,717.67){\rule{0.482pt}{0.400pt}}
\multiput(359.00,718.17)(1.000,-1.000){2}{\rule{0.241pt}{0.400pt}}
\put(361,716.67){\rule{0.723pt}{0.400pt}}
\multiput(361.00,717.17)(1.500,-1.000){2}{\rule{0.361pt}{0.400pt}}
\put(364,715.67){\rule{0.482pt}{0.400pt}}
\multiput(364.00,716.17)(1.000,-1.000){2}{\rule{0.241pt}{0.400pt}}
\put(366,714.67){\rule{0.482pt}{0.400pt}}
\multiput(366.00,715.17)(1.000,-1.000){2}{\rule{0.241pt}{0.400pt}}
\put(354.0,720.0){\rule[-0.200pt]{0.482pt}{0.400pt}}
\put(371,713.67){\rule{0.482pt}{0.400pt}}
\multiput(371.00,714.17)(1.000,-1.000){2}{\rule{0.241pt}{0.400pt}}
\put(373,712.67){\rule{0.723pt}{0.400pt}}
\multiput(373.00,713.17)(1.500,-1.000){2}{\rule{0.361pt}{0.400pt}}
\put(376,711.67){\rule{0.482pt}{0.400pt}}
\multiput(376.00,712.17)(1.000,-1.000){2}{\rule{0.241pt}{0.400pt}}
\put(378,710.67){\rule{0.723pt}{0.400pt}}
\multiput(378.00,711.17)(1.500,-1.000){2}{\rule{0.361pt}{0.400pt}}
\put(381,709.67){\rule{0.482pt}{0.400pt}}
\multiput(381.00,710.17)(1.000,-1.000){2}{\rule{0.241pt}{0.400pt}}
\put(383,708.67){\rule{0.723pt}{0.400pt}}
\multiput(383.00,709.17)(1.500,-1.000){2}{\rule{0.361pt}{0.400pt}}
\put(368.0,715.0){\rule[-0.200pt]{0.723pt}{0.400pt}}
\put(388,707.67){\rule{0.723pt}{0.400pt}}
\multiput(388.00,708.17)(1.500,-1.000){2}{\rule{0.361pt}{0.400pt}}
\put(391,706.67){\rule{0.482pt}{0.400pt}}
\multiput(391.00,707.17)(1.000,-1.000){2}{\rule{0.241pt}{0.400pt}}
\put(393,705.67){\rule{0.723pt}{0.400pt}}
\multiput(393.00,706.17)(1.500,-1.000){2}{\rule{0.361pt}{0.400pt}}
\put(396,704.67){\rule{0.482pt}{0.400pt}}
\multiput(396.00,705.17)(1.000,-1.000){2}{\rule{0.241pt}{0.400pt}}
\put(398,703.67){\rule{0.482pt}{0.400pt}}
\multiput(398.00,704.17)(1.000,-1.000){2}{\rule{0.241pt}{0.400pt}}
\put(386.0,709.0){\rule[-0.200pt]{0.482pt}{0.400pt}}
\put(403,702.67){\rule{0.482pt}{0.400pt}}
\multiput(403.00,703.17)(1.000,-1.000){2}{\rule{0.241pt}{0.400pt}}
\put(405,701.67){\rule{0.723pt}{0.400pt}}
\multiput(405.00,702.17)(1.500,-1.000){2}{\rule{0.361pt}{0.400pt}}
\put(408,700.67){\rule{0.482pt}{0.400pt}}
\multiput(408.00,701.17)(1.000,-1.000){2}{\rule{0.241pt}{0.400pt}}
\put(410,699.67){\rule{0.723pt}{0.400pt}}
\multiput(410.00,700.17)(1.500,-1.000){2}{\rule{0.361pt}{0.400pt}}
\put(413,698.67){\rule{0.482pt}{0.400pt}}
\multiput(413.00,699.17)(1.000,-1.000){2}{\rule{0.241pt}{0.400pt}}
\put(415,697.67){\rule{0.723pt}{0.400pt}}
\multiput(415.00,698.17)(1.500,-1.000){2}{\rule{0.361pt}{0.400pt}}
\put(400.0,704.0){\rule[-0.200pt]{0.723pt}{0.400pt}}
\put(420,696.67){\rule{0.723pt}{0.400pt}}
\multiput(420.00,697.17)(1.500,-1.000){2}{\rule{0.361pt}{0.400pt}}
\put(423,695.67){\rule{0.482pt}{0.400pt}}
\multiput(423.00,696.17)(1.000,-1.000){2}{\rule{0.241pt}{0.400pt}}
\put(425,694.67){\rule{0.723pt}{0.400pt}}
\multiput(425.00,695.17)(1.500,-1.000){2}{\rule{0.361pt}{0.400pt}}
\put(428,693.67){\rule{0.482pt}{0.400pt}}
\multiput(428.00,694.17)(1.000,-1.000){2}{\rule{0.241pt}{0.400pt}}
\put(430,692.67){\rule{0.482pt}{0.400pt}}
\multiput(430.00,693.17)(1.000,-1.000){2}{\rule{0.241pt}{0.400pt}}
\put(432,691.67){\rule{0.723pt}{0.400pt}}
\multiput(432.00,692.17)(1.500,-1.000){2}{\rule{0.361pt}{0.400pt}}
\put(435,690.67){\rule{0.482pt}{0.400pt}}
\multiput(435.00,691.17)(1.000,-1.000){2}{\rule{0.241pt}{0.400pt}}
\put(418.0,698.0){\rule[-0.200pt]{0.482pt}{0.400pt}}
\put(440,689.67){\rule{0.482pt}{0.400pt}}
\multiput(440.00,690.17)(1.000,-1.000){2}{\rule{0.241pt}{0.400pt}}
\put(442,688.67){\rule{0.723pt}{0.400pt}}
\multiput(442.00,689.17)(1.500,-1.000){2}{\rule{0.361pt}{0.400pt}}
\put(445,687.67){\rule{0.482pt}{0.400pt}}
\multiput(445.00,688.17)(1.000,-1.000){2}{\rule{0.241pt}{0.400pt}}
\put(447,686.67){\rule{0.723pt}{0.400pt}}
\multiput(447.00,687.17)(1.500,-1.000){2}{\rule{0.361pt}{0.400pt}}
\put(450,685.67){\rule{0.482pt}{0.400pt}}
\multiput(450.00,686.17)(1.000,-1.000){2}{\rule{0.241pt}{0.400pt}}
\put(452,684.67){\rule{0.723pt}{0.400pt}}
\multiput(452.00,685.17)(1.500,-1.000){2}{\rule{0.361pt}{0.400pt}}
\put(455,683.67){\rule{0.482pt}{0.400pt}}
\multiput(455.00,684.17)(1.000,-1.000){2}{\rule{0.241pt}{0.400pt}}
\put(437.0,691.0){\rule[-0.200pt]{0.723pt}{0.400pt}}
\put(460,682.67){\rule{0.482pt}{0.400pt}}
\multiput(460.00,683.17)(1.000,-1.000){2}{\rule{0.241pt}{0.400pt}}
\put(462,681.67){\rule{0.482pt}{0.400pt}}
\multiput(462.00,682.17)(1.000,-1.000){2}{\rule{0.241pt}{0.400pt}}
\put(464,680.67){\rule{0.723pt}{0.400pt}}
\multiput(464.00,681.17)(1.500,-1.000){2}{\rule{0.361pt}{0.400pt}}
\put(467,679.67){\rule{0.482pt}{0.400pt}}
\multiput(467.00,680.17)(1.000,-1.000){2}{\rule{0.241pt}{0.400pt}}
\put(469,678.67){\rule{0.723pt}{0.400pt}}
\multiput(469.00,679.17)(1.500,-1.000){2}{\rule{0.361pt}{0.400pt}}
\put(472,677.67){\rule{0.482pt}{0.400pt}}
\multiput(472.00,678.17)(1.000,-1.000){2}{\rule{0.241pt}{0.400pt}}
\put(474,676.67){\rule{0.723pt}{0.400pt}}
\multiput(474.00,677.17)(1.500,-1.000){2}{\rule{0.361pt}{0.400pt}}
\put(477,675.67){\rule{0.482pt}{0.400pt}}
\multiput(477.00,676.17)(1.000,-1.000){2}{\rule{0.241pt}{0.400pt}}
\put(457.0,684.0){\rule[-0.200pt]{0.723pt}{0.400pt}}
\put(482,674.67){\rule{0.482pt}{0.400pt}}
\multiput(482.00,675.17)(1.000,-1.000){2}{\rule{0.241pt}{0.400pt}}
\put(484,673.67){\rule{0.723pt}{0.400pt}}
\multiput(484.00,674.17)(1.500,-1.000){2}{\rule{0.361pt}{0.400pt}}
\put(487,672.67){\rule{0.482pt}{0.400pt}}
\multiput(487.00,673.17)(1.000,-1.000){2}{\rule{0.241pt}{0.400pt}}
\put(489,671.67){\rule{0.723pt}{0.400pt}}
\multiput(489.00,672.17)(1.500,-1.000){2}{\rule{0.361pt}{0.400pt}}
\put(492,670.67){\rule{0.482pt}{0.400pt}}
\multiput(492.00,671.17)(1.000,-1.000){2}{\rule{0.241pt}{0.400pt}}
\put(494,669.67){\rule{0.482pt}{0.400pt}}
\multiput(494.00,670.17)(1.000,-1.000){2}{\rule{0.241pt}{0.400pt}}
\put(496,668.67){\rule{0.723pt}{0.400pt}}
\multiput(496.00,669.17)(1.500,-1.000){2}{\rule{0.361pt}{0.400pt}}
\put(499,667.67){\rule{0.482pt}{0.400pt}}
\multiput(499.00,668.17)(1.000,-1.000){2}{\rule{0.241pt}{0.400pt}}
\put(501,666.67){\rule{0.723pt}{0.400pt}}
\multiput(501.00,667.17)(1.500,-1.000){2}{\rule{0.361pt}{0.400pt}}
\put(479.0,676.0){\rule[-0.200pt]{0.723pt}{0.400pt}}
\put(506,665.67){\rule{0.723pt}{0.400pt}}
\multiput(506.00,666.17)(1.500,-1.000){2}{\rule{0.361pt}{0.400pt}}
\put(509,664.67){\rule{0.482pt}{0.400pt}}
\multiput(509.00,665.17)(1.000,-1.000){2}{\rule{0.241pt}{0.400pt}}
\put(511,663.67){\rule{0.723pt}{0.400pt}}
\multiput(511.00,664.17)(1.500,-1.000){2}{\rule{0.361pt}{0.400pt}}
\put(514,662.67){\rule{0.482pt}{0.400pt}}
\multiput(514.00,663.17)(1.000,-1.000){2}{\rule{0.241pt}{0.400pt}}
\put(516,661.67){\rule{0.723pt}{0.400pt}}
\multiput(516.00,662.17)(1.500,-1.000){2}{\rule{0.361pt}{0.400pt}}
\put(519,660.67){\rule{0.482pt}{0.400pt}}
\multiput(519.00,661.17)(1.000,-1.000){2}{\rule{0.241pt}{0.400pt}}
\put(521,659.67){\rule{0.723pt}{0.400pt}}
\multiput(521.00,660.17)(1.500,-1.000){2}{\rule{0.361pt}{0.400pt}}
\put(524,658.67){\rule{0.482pt}{0.400pt}}
\multiput(524.00,659.17)(1.000,-1.000){2}{\rule{0.241pt}{0.400pt}}
\put(526,657.67){\rule{0.482pt}{0.400pt}}
\multiput(526.00,658.17)(1.000,-1.000){2}{\rule{0.241pt}{0.400pt}}
\put(528,656.67){\rule{0.723pt}{0.400pt}}
\multiput(528.00,657.17)(1.500,-1.000){2}{\rule{0.361pt}{0.400pt}}
\put(531,655.67){\rule{0.482pt}{0.400pt}}
\multiput(531.00,656.17)(1.000,-1.000){2}{\rule{0.241pt}{0.400pt}}
\put(504.0,667.0){\rule[-0.200pt]{0.482pt}{0.400pt}}
\put(536,654.67){\rule{0.482pt}{0.400pt}}
\multiput(536.00,655.17)(1.000,-1.000){2}{\rule{0.241pt}{0.400pt}}
\put(538,653.67){\rule{0.723pt}{0.400pt}}
\multiput(538.00,654.17)(1.500,-1.000){2}{\rule{0.361pt}{0.400pt}}
\put(541,652.67){\rule{0.482pt}{0.400pt}}
\multiput(541.00,653.17)(1.000,-1.000){2}{\rule{0.241pt}{0.400pt}}
\put(543,651.67){\rule{0.723pt}{0.400pt}}
\multiput(543.00,652.17)(1.500,-1.000){2}{\rule{0.361pt}{0.400pt}}
\put(546,650.67){\rule{0.482pt}{0.400pt}}
\multiput(546.00,651.17)(1.000,-1.000){2}{\rule{0.241pt}{0.400pt}}
\put(548,649.67){\rule{0.723pt}{0.400pt}}
\multiput(548.00,650.17)(1.500,-1.000){2}{\rule{0.361pt}{0.400pt}}
\put(551,648.67){\rule{0.482pt}{0.400pt}}
\multiput(551.00,649.17)(1.000,-1.000){2}{\rule{0.241pt}{0.400pt}}
\put(553,647.67){\rule{0.723pt}{0.400pt}}
\multiput(553.00,648.17)(1.500,-1.000){2}{\rule{0.361pt}{0.400pt}}
\put(556,646.67){\rule{0.482pt}{0.400pt}}
\multiput(556.00,647.17)(1.000,-1.000){2}{\rule{0.241pt}{0.400pt}}
\put(558,645.67){\rule{0.482pt}{0.400pt}}
\multiput(558.00,646.17)(1.000,-1.000){2}{\rule{0.241pt}{0.400pt}}
\put(560,644.67){\rule{0.723pt}{0.400pt}}
\multiput(560.00,645.17)(1.500,-1.000){2}{\rule{0.361pt}{0.400pt}}
\put(563,643.67){\rule{0.482pt}{0.400pt}}
\multiput(563.00,644.17)(1.000,-1.000){2}{\rule{0.241pt}{0.400pt}}
\put(565,642.67){\rule{0.723pt}{0.400pt}}
\multiput(565.00,643.17)(1.500,-1.000){2}{\rule{0.361pt}{0.400pt}}
\put(568,641.67){\rule{0.482pt}{0.400pt}}
\multiput(568.00,642.17)(1.000,-1.000){2}{\rule{0.241pt}{0.400pt}}
\put(533.0,656.0){\rule[-0.200pt]{0.723pt}{0.400pt}}
\put(573,640.67){\rule{0.482pt}{0.400pt}}
\multiput(573.00,641.17)(1.000,-1.000){2}{\rule{0.241pt}{0.400pt}}
\put(575,639.67){\rule{0.723pt}{0.400pt}}
\multiput(575.00,640.17)(1.500,-1.000){2}{\rule{0.361pt}{0.400pt}}
\put(578,638.67){\rule{0.482pt}{0.400pt}}
\multiput(578.00,639.17)(1.000,-1.000){2}{\rule{0.241pt}{0.400pt}}
\put(580,637.67){\rule{0.723pt}{0.400pt}}
\multiput(580.00,638.17)(1.500,-1.000){2}{\rule{0.361pt}{0.400pt}}
\put(583,636.67){\rule{0.482pt}{0.400pt}}
\multiput(583.00,637.17)(1.000,-1.000){2}{\rule{0.241pt}{0.400pt}}
\put(585,635.67){\rule{0.723pt}{0.400pt}}
\multiput(585.00,636.17)(1.500,-1.000){2}{\rule{0.361pt}{0.400pt}}
\put(588,634.67){\rule{0.482pt}{0.400pt}}
\multiput(588.00,635.17)(1.000,-1.000){2}{\rule{0.241pt}{0.400pt}}
\put(590,633.67){\rule{0.482pt}{0.400pt}}
\multiput(590.00,634.17)(1.000,-1.000){2}{\rule{0.241pt}{0.400pt}}
\put(592,632.67){\rule{0.723pt}{0.400pt}}
\multiput(592.00,633.17)(1.500,-1.000){2}{\rule{0.361pt}{0.400pt}}
\put(595,631.67){\rule{0.482pt}{0.400pt}}
\multiput(595.00,632.17)(1.000,-1.000){2}{\rule{0.241pt}{0.400pt}}
\put(597,630.67){\rule{0.723pt}{0.400pt}}
\multiput(597.00,631.17)(1.500,-1.000){2}{\rule{0.361pt}{0.400pt}}
\put(600,629.67){\rule{0.482pt}{0.400pt}}
\multiput(600.00,630.17)(1.000,-1.000){2}{\rule{0.241pt}{0.400pt}}
\put(602,628.67){\rule{0.723pt}{0.400pt}}
\multiput(602.00,629.17)(1.500,-1.000){2}{\rule{0.361pt}{0.400pt}}
\put(605,627.67){\rule{0.482pt}{0.400pt}}
\multiput(605.00,628.17)(1.000,-1.000){2}{\rule{0.241pt}{0.400pt}}
\put(607,626.67){\rule{0.723pt}{0.400pt}}
\multiput(607.00,627.17)(1.500,-1.000){2}{\rule{0.361pt}{0.400pt}}
\put(610,625.67){\rule{0.482pt}{0.400pt}}
\multiput(610.00,626.17)(1.000,-1.000){2}{\rule{0.241pt}{0.400pt}}
\put(612,624.67){\rule{0.723pt}{0.400pt}}
\multiput(612.00,625.17)(1.500,-1.000){2}{\rule{0.361pt}{0.400pt}}
\put(615,623.67){\rule{0.482pt}{0.400pt}}
\multiput(615.00,624.17)(1.000,-1.000){2}{\rule{0.241pt}{0.400pt}}
\put(617,622.67){\rule{0.723pt}{0.400pt}}
\multiput(617.00,623.17)(1.500,-1.000){2}{\rule{0.361pt}{0.400pt}}
\put(620,621.67){\rule{0.482pt}{0.400pt}}
\multiput(620.00,622.17)(1.000,-1.000){2}{\rule{0.241pt}{0.400pt}}
\put(622,620.67){\rule{0.482pt}{0.400pt}}
\multiput(622.00,621.17)(1.000,-1.000){2}{\rule{0.241pt}{0.400pt}}
\put(624,619.67){\rule{0.723pt}{0.400pt}}
\multiput(624.00,620.17)(1.500,-1.000){2}{\rule{0.361pt}{0.400pt}}
\put(627,618.67){\rule{0.482pt}{0.400pt}}
\multiput(627.00,619.17)(1.000,-1.000){2}{\rule{0.241pt}{0.400pt}}
\put(629,617.67){\rule{0.723pt}{0.400pt}}
\multiput(629.00,618.17)(1.500,-1.000){2}{\rule{0.361pt}{0.400pt}}
\put(632,616.67){\rule{0.482pt}{0.400pt}}
\multiput(632.00,617.17)(1.000,-1.000){2}{\rule{0.241pt}{0.400pt}}
\put(634,615.67){\rule{0.723pt}{0.400pt}}
\multiput(634.00,616.17)(1.500,-1.000){2}{\rule{0.361pt}{0.400pt}}
\put(637,614.67){\rule{0.482pt}{0.400pt}}
\multiput(637.00,615.17)(1.000,-1.000){2}{\rule{0.241pt}{0.400pt}}
\put(639,613.67){\rule{0.723pt}{0.400pt}}
\multiput(639.00,614.17)(1.500,-1.000){2}{\rule{0.361pt}{0.400pt}}
\put(642,612.67){\rule{0.482pt}{0.400pt}}
\multiput(642.00,613.17)(1.000,-1.000){2}{\rule{0.241pt}{0.400pt}}
\put(644,611.67){\rule{0.723pt}{0.400pt}}
\multiput(644.00,612.17)(1.500,-1.000){2}{\rule{0.361pt}{0.400pt}}
\put(647,610.67){\rule{0.482pt}{0.400pt}}
\multiput(647.00,611.17)(1.000,-1.000){2}{\rule{0.241pt}{0.400pt}}
\put(649,609.67){\rule{0.723pt}{0.400pt}}
\multiput(649.00,610.17)(1.500,-1.000){2}{\rule{0.361pt}{0.400pt}}
\put(652,608.67){\rule{0.482pt}{0.400pt}}
\multiput(652.00,609.17)(1.000,-1.000){2}{\rule{0.241pt}{0.400pt}}
\put(654,607.67){\rule{0.482pt}{0.400pt}}
\multiput(654.00,608.17)(1.000,-1.000){2}{\rule{0.241pt}{0.400pt}}
\put(656,606.67){\rule{0.723pt}{0.400pt}}
\multiput(656.00,607.17)(1.500,-1.000){2}{\rule{0.361pt}{0.400pt}}
\put(659,605.67){\rule{0.482pt}{0.400pt}}
\multiput(659.00,606.17)(1.000,-1.000){2}{\rule{0.241pt}{0.400pt}}
\put(661,604.67){\rule{0.723pt}{0.400pt}}
\multiput(661.00,605.17)(1.500,-1.000){2}{\rule{0.361pt}{0.400pt}}
\put(664,603.67){\rule{0.482pt}{0.400pt}}
\multiput(664.00,604.17)(1.000,-1.000){2}{\rule{0.241pt}{0.400pt}}
\put(666,602.67){\rule{0.723pt}{0.400pt}}
\multiput(666.00,603.17)(1.500,-1.000){2}{\rule{0.361pt}{0.400pt}}
\put(669,601.67){\rule{0.482pt}{0.400pt}}
\multiput(669.00,602.17)(1.000,-1.000){2}{\rule{0.241pt}{0.400pt}}
\put(671,600.67){\rule{0.723pt}{0.400pt}}
\multiput(671.00,601.17)(1.500,-1.000){2}{\rule{0.361pt}{0.400pt}}
\put(674,599.67){\rule{0.482pt}{0.400pt}}
\multiput(674.00,600.17)(1.000,-1.000){2}{\rule{0.241pt}{0.400pt}}
\put(676,598.67){\rule{0.723pt}{0.400pt}}
\multiput(676.00,599.17)(1.500,-1.000){2}{\rule{0.361pt}{0.400pt}}
\put(679,597.67){\rule{0.482pt}{0.400pt}}
\multiput(679.00,598.17)(1.000,-1.000){2}{\rule{0.241pt}{0.400pt}}
\put(681,596.67){\rule{0.482pt}{0.400pt}}
\multiput(681.00,597.17)(1.000,-1.000){2}{\rule{0.241pt}{0.400pt}}
\put(683,595.67){\rule{0.723pt}{0.400pt}}
\multiput(683.00,596.17)(1.500,-1.000){2}{\rule{0.361pt}{0.400pt}}
\put(686,594.67){\rule{0.482pt}{0.400pt}}
\multiput(686.00,595.17)(1.000,-1.000){2}{\rule{0.241pt}{0.400pt}}
\put(688,593.67){\rule{0.723pt}{0.400pt}}
\multiput(688.00,594.17)(1.500,-1.000){2}{\rule{0.361pt}{0.400pt}}
\put(691,592.67){\rule{0.482pt}{0.400pt}}
\multiput(691.00,593.17)(1.000,-1.000){2}{\rule{0.241pt}{0.400pt}}
\put(693,591.67){\rule{0.723pt}{0.400pt}}
\multiput(693.00,592.17)(1.500,-1.000){2}{\rule{0.361pt}{0.400pt}}
\put(696,590.67){\rule{0.482pt}{0.400pt}}
\multiput(696.00,591.17)(1.000,-1.000){2}{\rule{0.241pt}{0.400pt}}
\put(698,589.67){\rule{0.723pt}{0.400pt}}
\multiput(698.00,590.17)(1.500,-1.000){2}{\rule{0.361pt}{0.400pt}}
\put(701,588.67){\rule{0.482pt}{0.400pt}}
\multiput(701.00,589.17)(1.000,-1.000){2}{\rule{0.241pt}{0.400pt}}
\put(703,587.67){\rule{0.723pt}{0.400pt}}
\multiput(703.00,588.17)(1.500,-1.000){2}{\rule{0.361pt}{0.400pt}}
\put(706,586.67){\rule{0.482pt}{0.400pt}}
\multiput(706.00,587.17)(1.000,-1.000){2}{\rule{0.241pt}{0.400pt}}
\put(708,585.67){\rule{0.723pt}{0.400pt}}
\multiput(708.00,586.17)(1.500,-1.000){2}{\rule{0.361pt}{0.400pt}}
\put(711,584.67){\rule{0.482pt}{0.400pt}}
\multiput(711.00,585.17)(1.000,-1.000){2}{\rule{0.241pt}{0.400pt}}
\put(713,583.67){\rule{0.482pt}{0.400pt}}
\multiput(713.00,584.17)(1.000,-1.000){2}{\rule{0.241pt}{0.400pt}}
\put(715,582.67){\rule{0.723pt}{0.400pt}}
\multiput(715.00,583.17)(1.500,-1.000){2}{\rule{0.361pt}{0.400pt}}
\put(718,581.67){\rule{0.482pt}{0.400pt}}
\multiput(718.00,582.17)(1.000,-1.000){2}{\rule{0.241pt}{0.400pt}}
\put(720,580.67){\rule{0.723pt}{0.400pt}}
\multiput(720.00,581.17)(1.500,-1.000){2}{\rule{0.361pt}{0.400pt}}
\put(723,579.67){\rule{0.482pt}{0.400pt}}
\multiput(723.00,580.17)(1.000,-1.000){2}{\rule{0.241pt}{0.400pt}}
\put(725,578.67){\rule{0.723pt}{0.400pt}}
\multiput(725.00,579.17)(1.500,-1.000){2}{\rule{0.361pt}{0.400pt}}
\put(728,577.17){\rule{0.482pt}{0.400pt}}
\multiput(728.00,578.17)(1.000,-2.000){2}{\rule{0.241pt}{0.400pt}}
\put(730,575.67){\rule{0.723pt}{0.400pt}}
\multiput(730.00,576.17)(1.500,-1.000){2}{\rule{0.361pt}{0.400pt}}
\put(733,574.67){\rule{0.482pt}{0.400pt}}
\multiput(733.00,575.17)(1.000,-1.000){2}{\rule{0.241pt}{0.400pt}}
\put(735,573.67){\rule{0.723pt}{0.400pt}}
\multiput(735.00,574.17)(1.500,-1.000){2}{\rule{0.361pt}{0.400pt}}
\put(738,572.67){\rule{0.482pt}{0.400pt}}
\multiput(738.00,573.17)(1.000,-1.000){2}{\rule{0.241pt}{0.400pt}}
\put(740,571.67){\rule{0.723pt}{0.400pt}}
\multiput(740.00,572.17)(1.500,-1.000){2}{\rule{0.361pt}{0.400pt}}
\put(743,570.67){\rule{0.482pt}{0.400pt}}
\multiput(743.00,571.17)(1.000,-1.000){2}{\rule{0.241pt}{0.400pt}}
\put(745,569.67){\rule{0.482pt}{0.400pt}}
\multiput(745.00,570.17)(1.000,-1.000){2}{\rule{0.241pt}{0.400pt}}
\put(747,568.67){\rule{0.723pt}{0.400pt}}
\multiput(747.00,569.17)(1.500,-1.000){2}{\rule{0.361pt}{0.400pt}}
\put(750,567.67){\rule{0.482pt}{0.400pt}}
\multiput(750.00,568.17)(1.000,-1.000){2}{\rule{0.241pt}{0.400pt}}
\put(752,566.67){\rule{0.723pt}{0.400pt}}
\multiput(752.00,567.17)(1.500,-1.000){2}{\rule{0.361pt}{0.400pt}}
\put(755,565.67){\rule{0.482pt}{0.400pt}}
\multiput(755.00,566.17)(1.000,-1.000){2}{\rule{0.241pt}{0.400pt}}
\put(757,564.67){\rule{0.723pt}{0.400pt}}
\multiput(757.00,565.17)(1.500,-1.000){2}{\rule{0.361pt}{0.400pt}}
\put(760,563.67){\rule{0.482pt}{0.400pt}}
\multiput(760.00,564.17)(1.000,-1.000){2}{\rule{0.241pt}{0.400pt}}
\put(762,562.17){\rule{0.700pt}{0.400pt}}
\multiput(762.00,563.17)(1.547,-2.000){2}{\rule{0.350pt}{0.400pt}}
\put(765,560.67){\rule{0.482pt}{0.400pt}}
\multiput(765.00,561.17)(1.000,-1.000){2}{\rule{0.241pt}{0.400pt}}
\put(767,559.67){\rule{0.723pt}{0.400pt}}
\multiput(767.00,560.17)(1.500,-1.000){2}{\rule{0.361pt}{0.400pt}}
\put(770,558.67){\rule{0.482pt}{0.400pt}}
\multiput(770.00,559.17)(1.000,-1.000){2}{\rule{0.241pt}{0.400pt}}
\put(772,557.67){\rule{0.723pt}{0.400pt}}
\multiput(772.00,558.17)(1.500,-1.000){2}{\rule{0.361pt}{0.400pt}}
\put(775,556.67){\rule{0.482pt}{0.400pt}}
\multiput(775.00,557.17)(1.000,-1.000){2}{\rule{0.241pt}{0.400pt}}
\put(777,555.67){\rule{0.482pt}{0.400pt}}
\multiput(777.00,556.17)(1.000,-1.000){2}{\rule{0.241pt}{0.400pt}}
\put(779,554.67){\rule{0.723pt}{0.400pt}}
\multiput(779.00,555.17)(1.500,-1.000){2}{\rule{0.361pt}{0.400pt}}
\put(782,553.67){\rule{0.482pt}{0.400pt}}
\multiput(782.00,554.17)(1.000,-1.000){2}{\rule{0.241pt}{0.400pt}}
\put(784,552.67){\rule{0.723pt}{0.400pt}}
\multiput(784.00,553.17)(1.500,-1.000){2}{\rule{0.361pt}{0.400pt}}
\put(787,551.17){\rule{0.482pt}{0.400pt}}
\multiput(787.00,552.17)(1.000,-2.000){2}{\rule{0.241pt}{0.400pt}}
\put(789,549.67){\rule{0.723pt}{0.400pt}}
\multiput(789.00,550.17)(1.500,-1.000){2}{\rule{0.361pt}{0.400pt}}
\put(792,548.67){\rule{0.482pt}{0.400pt}}
\multiput(792.00,549.17)(1.000,-1.000){2}{\rule{0.241pt}{0.400pt}}
\put(794,547.67){\rule{0.723pt}{0.400pt}}
\multiput(794.00,548.17)(1.500,-1.000){2}{\rule{0.361pt}{0.400pt}}
\put(797,546.67){\rule{0.482pt}{0.400pt}}
\multiput(797.00,547.17)(1.000,-1.000){2}{\rule{0.241pt}{0.400pt}}
\put(799,545.67){\rule{0.723pt}{0.400pt}}
\multiput(799.00,546.17)(1.500,-1.000){2}{\rule{0.361pt}{0.400pt}}
\put(802,544.67){\rule{0.482pt}{0.400pt}}
\multiput(802.00,545.17)(1.000,-1.000){2}{\rule{0.241pt}{0.400pt}}
\put(804,543.67){\rule{0.723pt}{0.400pt}}
\multiput(804.00,544.17)(1.500,-1.000){2}{\rule{0.361pt}{0.400pt}}
\put(807,542.67){\rule{0.482pt}{0.400pt}}
\multiput(807.00,543.17)(1.000,-1.000){2}{\rule{0.241pt}{0.400pt}}
\put(809,541.17){\rule{0.482pt}{0.400pt}}
\multiput(809.00,542.17)(1.000,-2.000){2}{\rule{0.241pt}{0.400pt}}
\put(811,539.67){\rule{0.723pt}{0.400pt}}
\multiput(811.00,540.17)(1.500,-1.000){2}{\rule{0.361pt}{0.400pt}}
\put(814,538.67){\rule{0.482pt}{0.400pt}}
\multiput(814.00,539.17)(1.000,-1.000){2}{\rule{0.241pt}{0.400pt}}
\put(816,537.67){\rule{0.723pt}{0.400pt}}
\multiput(816.00,538.17)(1.500,-1.000){2}{\rule{0.361pt}{0.400pt}}
\put(819,536.67){\rule{0.482pt}{0.400pt}}
\multiput(819.00,537.17)(1.000,-1.000){2}{\rule{0.241pt}{0.400pt}}
\put(821,535.67){\rule{0.723pt}{0.400pt}}
\multiput(821.00,536.17)(1.500,-1.000){2}{\rule{0.361pt}{0.400pt}}
\put(824,534.67){\rule{0.482pt}{0.400pt}}
\multiput(824.00,535.17)(1.000,-1.000){2}{\rule{0.241pt}{0.400pt}}
\put(826,533.17){\rule{0.700pt}{0.400pt}}
\multiput(826.00,534.17)(1.547,-2.000){2}{\rule{0.350pt}{0.400pt}}
\put(829,531.67){\rule{0.482pt}{0.400pt}}
\multiput(829.00,532.17)(1.000,-1.000){2}{\rule{0.241pt}{0.400pt}}
\put(831,530.67){\rule{0.723pt}{0.400pt}}
\multiput(831.00,531.17)(1.500,-1.000){2}{\rule{0.361pt}{0.400pt}}
\put(834,529.67){\rule{0.482pt}{0.400pt}}
\multiput(834.00,530.17)(1.000,-1.000){2}{\rule{0.241pt}{0.400pt}}
\put(836,528.67){\rule{0.723pt}{0.400pt}}
\multiput(836.00,529.17)(1.500,-1.000){2}{\rule{0.361pt}{0.400pt}}
\put(839,527.67){\rule{0.482pt}{0.400pt}}
\multiput(839.00,528.17)(1.000,-1.000){2}{\rule{0.241pt}{0.400pt}}
\put(841,526.67){\rule{0.482pt}{0.400pt}}
\multiput(841.00,527.17)(1.000,-1.000){2}{\rule{0.241pt}{0.400pt}}
\put(843,525.17){\rule{0.700pt}{0.400pt}}
\multiput(843.00,526.17)(1.547,-2.000){2}{\rule{0.350pt}{0.400pt}}
\put(846,523.67){\rule{0.482pt}{0.400pt}}
\multiput(846.00,524.17)(1.000,-1.000){2}{\rule{0.241pt}{0.400pt}}
\put(848,522.67){\rule{0.723pt}{0.400pt}}
\multiput(848.00,523.17)(1.500,-1.000){2}{\rule{0.361pt}{0.400pt}}
\put(851,521.67){\rule{0.482pt}{0.400pt}}
\multiput(851.00,522.17)(1.000,-1.000){2}{\rule{0.241pt}{0.400pt}}
\put(853,520.67){\rule{0.723pt}{0.400pt}}
\multiput(853.00,521.17)(1.500,-1.000){2}{\rule{0.361pt}{0.400pt}}
\put(856,519.67){\rule{0.482pt}{0.400pt}}
\multiput(856.00,520.17)(1.000,-1.000){2}{\rule{0.241pt}{0.400pt}}
\put(858,518.17){\rule{0.700pt}{0.400pt}}
\multiput(858.00,519.17)(1.547,-2.000){2}{\rule{0.350pt}{0.400pt}}
\put(861,516.67){\rule{0.482pt}{0.400pt}}
\multiput(861.00,517.17)(1.000,-1.000){2}{\rule{0.241pt}{0.400pt}}
\put(863,515.67){\rule{0.723pt}{0.400pt}}
\multiput(863.00,516.17)(1.500,-1.000){2}{\rule{0.361pt}{0.400pt}}
\put(866,514.67){\rule{0.482pt}{0.400pt}}
\multiput(866.00,515.17)(1.000,-1.000){2}{\rule{0.241pt}{0.400pt}}
\put(868,513.67){\rule{0.723pt}{0.400pt}}
\multiput(868.00,514.17)(1.500,-1.000){2}{\rule{0.361pt}{0.400pt}}
\put(871,512.67){\rule{0.482pt}{0.400pt}}
\multiput(871.00,513.17)(1.000,-1.000){2}{\rule{0.241pt}{0.400pt}}
\put(873,511.17){\rule{0.482pt}{0.400pt}}
\multiput(873.00,512.17)(1.000,-2.000){2}{\rule{0.241pt}{0.400pt}}
\put(875,509.67){\rule{0.723pt}{0.400pt}}
\multiput(875.00,510.17)(1.500,-1.000){2}{\rule{0.361pt}{0.400pt}}
\put(878,508.67){\rule{0.482pt}{0.400pt}}
\multiput(878.00,509.17)(1.000,-1.000){2}{\rule{0.241pt}{0.400pt}}
\put(880,507.67){\rule{0.723pt}{0.400pt}}
\multiput(880.00,508.17)(1.500,-1.000){2}{\rule{0.361pt}{0.400pt}}
\put(883,506.67){\rule{0.482pt}{0.400pt}}
\multiput(883.00,507.17)(1.000,-1.000){2}{\rule{0.241pt}{0.400pt}}
\put(885,505.17){\rule{0.700pt}{0.400pt}}
\multiput(885.00,506.17)(1.547,-2.000){2}{\rule{0.350pt}{0.400pt}}
\put(888,503.67){\rule{0.482pt}{0.400pt}}
\multiput(888.00,504.17)(1.000,-1.000){2}{\rule{0.241pt}{0.400pt}}
\put(890,502.67){\rule{0.723pt}{0.400pt}}
\multiput(890.00,503.17)(1.500,-1.000){2}{\rule{0.361pt}{0.400pt}}
\put(893,501.67){\rule{0.482pt}{0.400pt}}
\multiput(893.00,502.17)(1.000,-1.000){2}{\rule{0.241pt}{0.400pt}}
\put(895,500.67){\rule{0.723pt}{0.400pt}}
\multiput(895.00,501.17)(1.500,-1.000){2}{\rule{0.361pt}{0.400pt}}
\put(898,499.17){\rule{0.482pt}{0.400pt}}
\multiput(898.00,500.17)(1.000,-2.000){2}{\rule{0.241pt}{0.400pt}}
\put(900,497.67){\rule{0.723pt}{0.400pt}}
\multiput(900.00,498.17)(1.500,-1.000){2}{\rule{0.361pt}{0.400pt}}
\put(903,496.67){\rule{0.482pt}{0.400pt}}
\multiput(903.00,497.17)(1.000,-1.000){2}{\rule{0.241pt}{0.400pt}}
\put(905,495.67){\rule{0.482pt}{0.400pt}}
\multiput(905.00,496.17)(1.000,-1.000){2}{\rule{0.241pt}{0.400pt}}
\put(907,494.67){\rule{0.723pt}{0.400pt}}
\multiput(907.00,495.17)(1.500,-1.000){2}{\rule{0.361pt}{0.400pt}}
\put(910,493.17){\rule{0.482pt}{0.400pt}}
\multiput(910.00,494.17)(1.000,-2.000){2}{\rule{0.241pt}{0.400pt}}
\put(912,491.67){\rule{0.723pt}{0.400pt}}
\multiput(912.00,492.17)(1.500,-1.000){2}{\rule{0.361pt}{0.400pt}}
\put(915,490.67){\rule{0.482pt}{0.400pt}}
\multiput(915.00,491.17)(1.000,-1.000){2}{\rule{0.241pt}{0.400pt}}
\put(917,489.67){\rule{0.723pt}{0.400pt}}
\multiput(917.00,490.17)(1.500,-1.000){2}{\rule{0.361pt}{0.400pt}}
\put(920,488.17){\rule{0.482pt}{0.400pt}}
\multiput(920.00,489.17)(1.000,-2.000){2}{\rule{0.241pt}{0.400pt}}
\put(922,486.67){\rule{0.723pt}{0.400pt}}
\multiput(922.00,487.17)(1.500,-1.000){2}{\rule{0.361pt}{0.400pt}}
\put(925,485.67){\rule{0.482pt}{0.400pt}}
\multiput(925.00,486.17)(1.000,-1.000){2}{\rule{0.241pt}{0.400pt}}
\put(927,484.67){\rule{0.723pt}{0.400pt}}
\multiput(927.00,485.17)(1.500,-1.000){2}{\rule{0.361pt}{0.400pt}}
\put(930,483.17){\rule{0.482pt}{0.400pt}}
\multiput(930.00,484.17)(1.000,-2.000){2}{\rule{0.241pt}{0.400pt}}
\put(932,481.67){\rule{0.723pt}{0.400pt}}
\multiput(932.00,482.17)(1.500,-1.000){2}{\rule{0.361pt}{0.400pt}}
\put(935,480.67){\rule{0.482pt}{0.400pt}}
\multiput(935.00,481.17)(1.000,-1.000){2}{\rule{0.241pt}{0.400pt}}
\put(937,479.67){\rule{0.482pt}{0.400pt}}
\multiput(937.00,480.17)(1.000,-1.000){2}{\rule{0.241pt}{0.400pt}}
\put(939,478.17){\rule{0.700pt}{0.400pt}}
\multiput(939.00,479.17)(1.547,-2.000){2}{\rule{0.350pt}{0.400pt}}
\put(942,476.67){\rule{0.482pt}{0.400pt}}
\multiput(942.00,477.17)(1.000,-1.000){2}{\rule{0.241pt}{0.400pt}}
\put(944,475.67){\rule{0.723pt}{0.400pt}}
\multiput(944.00,476.17)(1.500,-1.000){2}{\rule{0.361pt}{0.400pt}}
\put(947,474.67){\rule{0.482pt}{0.400pt}}
\multiput(947.00,475.17)(1.000,-1.000){2}{\rule{0.241pt}{0.400pt}}
\put(949,473.17){\rule{0.700pt}{0.400pt}}
\multiput(949.00,474.17)(1.547,-2.000){2}{\rule{0.350pt}{0.400pt}}
\put(952,471.67){\rule{0.482pt}{0.400pt}}
\multiput(952.00,472.17)(1.000,-1.000){2}{\rule{0.241pt}{0.400pt}}
\put(954,470.67){\rule{0.723pt}{0.400pt}}
\multiput(954.00,471.17)(1.500,-1.000){2}{\rule{0.361pt}{0.400pt}}
\put(957,469.67){\rule{0.482pt}{0.400pt}}
\multiput(957.00,470.17)(1.000,-1.000){2}{\rule{0.241pt}{0.400pt}}
\put(959,468.17){\rule{0.700pt}{0.400pt}}
\multiput(959.00,469.17)(1.547,-2.000){2}{\rule{0.350pt}{0.400pt}}
\put(962,466.67){\rule{0.482pt}{0.400pt}}
\multiput(962.00,467.17)(1.000,-1.000){2}{\rule{0.241pt}{0.400pt}}
\put(964,465.67){\rule{0.723pt}{0.400pt}}
\multiput(964.00,466.17)(1.500,-1.000){2}{\rule{0.361pt}{0.400pt}}
\put(967,464.67){\rule{0.482pt}{0.400pt}}
\multiput(967.00,465.17)(1.000,-1.000){2}{\rule{0.241pt}{0.400pt}}
\put(969,463.17){\rule{0.482pt}{0.400pt}}
\multiput(969.00,464.17)(1.000,-2.000){2}{\rule{0.241pt}{0.400pt}}
\put(971,461.67){\rule{0.723pt}{0.400pt}}
\multiput(971.00,462.17)(1.500,-1.000){2}{\rule{0.361pt}{0.400pt}}
\put(974,460.67){\rule{0.482pt}{0.400pt}}
\multiput(974.00,461.17)(1.000,-1.000){2}{\rule{0.241pt}{0.400pt}}
\put(976,459.17){\rule{0.700pt}{0.400pt}}
\multiput(976.00,460.17)(1.547,-2.000){2}{\rule{0.350pt}{0.400pt}}
\put(979,457.67){\rule{0.482pt}{0.400pt}}
\multiput(979.00,458.17)(1.000,-1.000){2}{\rule{0.241pt}{0.400pt}}
\put(981,456.67){\rule{0.723pt}{0.400pt}}
\multiput(981.00,457.17)(1.500,-1.000){2}{\rule{0.361pt}{0.400pt}}
\put(984,455.67){\rule{0.482pt}{0.400pt}}
\multiput(984.00,456.17)(1.000,-1.000){2}{\rule{0.241pt}{0.400pt}}
\put(986,454.17){\rule{0.700pt}{0.400pt}}
\multiput(986.00,455.17)(1.547,-2.000){2}{\rule{0.350pt}{0.400pt}}
\put(989,452.67){\rule{0.482pt}{0.400pt}}
\multiput(989.00,453.17)(1.000,-1.000){2}{\rule{0.241pt}{0.400pt}}
\put(991,451.67){\rule{0.723pt}{0.400pt}}
\multiput(991.00,452.17)(1.500,-1.000){2}{\rule{0.361pt}{0.400pt}}
\put(994,450.17){\rule{0.482pt}{0.400pt}}
\multiput(994.00,451.17)(1.000,-2.000){2}{\rule{0.241pt}{0.400pt}}
\put(996,448.67){\rule{0.482pt}{0.400pt}}
\multiput(996.00,449.17)(1.000,-1.000){2}{\rule{0.241pt}{0.400pt}}
\put(998,447.67){\rule{0.723pt}{0.400pt}}
\multiput(998.00,448.17)(1.500,-1.000){2}{\rule{0.361pt}{0.400pt}}
\put(1001,446.17){\rule{0.482pt}{0.400pt}}
\multiput(1001.00,447.17)(1.000,-2.000){2}{\rule{0.241pt}{0.400pt}}
\put(1003,444.67){\rule{0.723pt}{0.400pt}}
\multiput(1003.00,445.17)(1.500,-1.000){2}{\rule{0.361pt}{0.400pt}}
\put(1006,443.67){\rule{0.482pt}{0.400pt}}
\multiput(1006.00,444.17)(1.000,-1.000){2}{\rule{0.241pt}{0.400pt}}
\put(1008,442.17){\rule{0.700pt}{0.400pt}}
\multiput(1008.00,443.17)(1.547,-2.000){2}{\rule{0.350pt}{0.400pt}}
\put(1011,440.67){\rule{0.482pt}{0.400pt}}
\multiput(1011.00,441.17)(1.000,-1.000){2}{\rule{0.241pt}{0.400pt}}
\put(1013,439.67){\rule{0.723pt}{0.400pt}}
\multiput(1013.00,440.17)(1.500,-1.000){2}{\rule{0.361pt}{0.400pt}}
\put(1016,438.67){\rule{0.482pt}{0.400pt}}
\multiput(1016.00,439.17)(1.000,-1.000){2}{\rule{0.241pt}{0.400pt}}
\put(1018,437.17){\rule{0.700pt}{0.400pt}}
\multiput(1018.00,438.17)(1.547,-2.000){2}{\rule{0.350pt}{0.400pt}}
\put(1021,435.67){\rule{0.482pt}{0.400pt}}
\multiput(1021.00,436.17)(1.000,-1.000){2}{\rule{0.241pt}{0.400pt}}
\put(1023,434.67){\rule{0.723pt}{0.400pt}}
\multiput(1023.00,435.17)(1.500,-1.000){2}{\rule{0.361pt}{0.400pt}}
\put(1026,433.17){\rule{0.482pt}{0.400pt}}
\multiput(1026.00,434.17)(1.000,-2.000){2}{\rule{0.241pt}{0.400pt}}
\put(1028,431.67){\rule{0.482pt}{0.400pt}}
\multiput(1028.00,432.17)(1.000,-1.000){2}{\rule{0.241pt}{0.400pt}}
\put(1030,430.67){\rule{0.723pt}{0.400pt}}
\multiput(1030.00,431.17)(1.500,-1.000){2}{\rule{0.361pt}{0.400pt}}
\put(1033,429.17){\rule{0.482pt}{0.400pt}}
\multiput(1033.00,430.17)(1.000,-2.000){2}{\rule{0.241pt}{0.400pt}}
\put(1035,427.67){\rule{0.723pt}{0.400pt}}
\multiput(1035.00,428.17)(1.500,-1.000){2}{\rule{0.361pt}{0.400pt}}
\put(1038,426.17){\rule{0.482pt}{0.400pt}}
\multiput(1038.00,427.17)(1.000,-2.000){2}{\rule{0.241pt}{0.400pt}}
\put(1040,424.67){\rule{0.723pt}{0.400pt}}
\multiput(1040.00,425.17)(1.500,-1.000){2}{\rule{0.361pt}{0.400pt}}
\put(1043,423.67){\rule{0.482pt}{0.400pt}}
\multiput(1043.00,424.17)(1.000,-1.000){2}{\rule{0.241pt}{0.400pt}}
\put(1045,422.17){\rule{0.700pt}{0.400pt}}
\multiput(1045.00,423.17)(1.547,-2.000){2}{\rule{0.350pt}{0.400pt}}
\put(1048,420.67){\rule{0.482pt}{0.400pt}}
\multiput(1048.00,421.17)(1.000,-1.000){2}{\rule{0.241pt}{0.400pt}}
\put(1050,419.67){\rule{0.723pt}{0.400pt}}
\multiput(1050.00,420.17)(1.500,-1.000){2}{\rule{0.361pt}{0.400pt}}
\put(1053,418.17){\rule{0.482pt}{0.400pt}}
\multiput(1053.00,419.17)(1.000,-2.000){2}{\rule{0.241pt}{0.400pt}}
\put(1055,416.67){\rule{0.723pt}{0.400pt}}
\multiput(1055.00,417.17)(1.500,-1.000){2}{\rule{0.361pt}{0.400pt}}
\put(1058,415.67){\rule{0.482pt}{0.400pt}}
\multiput(1058.00,416.17)(1.000,-1.000){2}{\rule{0.241pt}{0.400pt}}
\put(1060,414.17){\rule{0.482pt}{0.400pt}}
\multiput(1060.00,415.17)(1.000,-2.000){2}{\rule{0.241pt}{0.400pt}}
\put(1062,412.67){\rule{0.723pt}{0.400pt}}
\multiput(1062.00,413.17)(1.500,-1.000){2}{\rule{0.361pt}{0.400pt}}
\put(1065,411.17){\rule{0.482pt}{0.400pt}}
\multiput(1065.00,412.17)(1.000,-2.000){2}{\rule{0.241pt}{0.400pt}}
\put(1067,409.67){\rule{0.723pt}{0.400pt}}
\multiput(1067.00,410.17)(1.500,-1.000){2}{\rule{0.361pt}{0.400pt}}
\put(1070,408.67){\rule{0.482pt}{0.400pt}}
\multiput(1070.00,409.17)(1.000,-1.000){2}{\rule{0.241pt}{0.400pt}}
\put(1072,407.17){\rule{0.700pt}{0.400pt}}
\multiput(1072.00,408.17)(1.547,-2.000){2}{\rule{0.350pt}{0.400pt}}
\put(1075,405.67){\rule{0.482pt}{0.400pt}}
\multiput(1075.00,406.17)(1.000,-1.000){2}{\rule{0.241pt}{0.400pt}}
\put(1077,404.67){\rule{0.723pt}{0.400pt}}
\multiput(1077.00,405.17)(1.500,-1.000){2}{\rule{0.361pt}{0.400pt}}
\put(1080,403.17){\rule{0.482pt}{0.400pt}}
\multiput(1080.00,404.17)(1.000,-2.000){2}{\rule{0.241pt}{0.400pt}}
\put(1082,401.67){\rule{0.723pt}{0.400pt}}
\multiput(1082.00,402.17)(1.500,-1.000){2}{\rule{0.361pt}{0.400pt}}
\put(1085,400.17){\rule{0.482pt}{0.400pt}}
\multiput(1085.00,401.17)(1.000,-2.000){2}{\rule{0.241pt}{0.400pt}}
\put(1087,398.67){\rule{0.723pt}{0.400pt}}
\multiput(1087.00,399.17)(1.500,-1.000){2}{\rule{0.361pt}{0.400pt}}
\put(1090,397.67){\rule{0.482pt}{0.400pt}}
\multiput(1090.00,398.17)(1.000,-1.000){2}{\rule{0.241pt}{0.400pt}}
\put(1092,396.17){\rule{0.482pt}{0.400pt}}
\multiput(1092.00,397.17)(1.000,-2.000){2}{\rule{0.241pt}{0.400pt}}
\put(1094,394.67){\rule{0.723pt}{0.400pt}}
\multiput(1094.00,395.17)(1.500,-1.000){2}{\rule{0.361pt}{0.400pt}}
\put(1097,393.17){\rule{0.482pt}{0.400pt}}
\multiput(1097.00,394.17)(1.000,-2.000){2}{\rule{0.241pt}{0.400pt}}
\put(1099,391.67){\rule{0.723pt}{0.400pt}}
\multiput(1099.00,392.17)(1.500,-1.000){2}{\rule{0.361pt}{0.400pt}}
\put(1102,390.17){\rule{0.482pt}{0.400pt}}
\multiput(1102.00,391.17)(1.000,-2.000){2}{\rule{0.241pt}{0.400pt}}
\put(1104,388.67){\rule{0.723pt}{0.400pt}}
\multiput(1104.00,389.17)(1.500,-1.000){2}{\rule{0.361pt}{0.400pt}}
\put(1107,387.67){\rule{0.482pt}{0.400pt}}
\multiput(1107.00,388.17)(1.000,-1.000){2}{\rule{0.241pt}{0.400pt}}
\put(1109,386.17){\rule{0.700pt}{0.400pt}}
\multiput(1109.00,387.17)(1.547,-2.000){2}{\rule{0.350pt}{0.400pt}}
\put(1112,384.67){\rule{0.482pt}{0.400pt}}
\multiput(1112.00,385.17)(1.000,-1.000){2}{\rule{0.241pt}{0.400pt}}
\put(1114,383.17){\rule{0.700pt}{0.400pt}}
\multiput(1114.00,384.17)(1.547,-2.000){2}{\rule{0.350pt}{0.400pt}}
\put(1117,381.67){\rule{0.482pt}{0.400pt}}
\multiput(1117.00,382.17)(1.000,-1.000){2}{\rule{0.241pt}{0.400pt}}
\put(1119,380.17){\rule{0.700pt}{0.400pt}}
\multiput(1119.00,381.17)(1.547,-2.000){2}{\rule{0.350pt}{0.400pt}}
\put(1122,378.67){\rule{0.482pt}{0.400pt}}
\multiput(1122.00,379.17)(1.000,-1.000){2}{\rule{0.241pt}{0.400pt}}
\put(1124,377.67){\rule{0.482pt}{0.400pt}}
\multiput(1124.00,378.17)(1.000,-1.000){2}{\rule{0.241pt}{0.400pt}}
\put(1126,376.17){\rule{0.700pt}{0.400pt}}
\multiput(1126.00,377.17)(1.547,-2.000){2}{\rule{0.350pt}{0.400pt}}
\put(1129,374.67){\rule{0.482pt}{0.400pt}}
\multiput(1129.00,375.17)(1.000,-1.000){2}{\rule{0.241pt}{0.400pt}}
\put(1131,373.17){\rule{0.700pt}{0.400pt}}
\multiput(1131.00,374.17)(1.547,-2.000){2}{\rule{0.350pt}{0.400pt}}
\put(1134,371.67){\rule{0.482pt}{0.400pt}}
\multiput(1134.00,372.17)(1.000,-1.000){2}{\rule{0.241pt}{0.400pt}}
\put(1136,370.17){\rule{0.700pt}{0.400pt}}
\multiput(1136.00,371.17)(1.547,-2.000){2}{\rule{0.350pt}{0.400pt}}
\put(1139,368.67){\rule{0.482pt}{0.400pt}}
\multiput(1139.00,369.17)(1.000,-1.000){2}{\rule{0.241pt}{0.400pt}}
\put(1141,367.17){\rule{0.700pt}{0.400pt}}
\multiput(1141.00,368.17)(1.547,-2.000){2}{\rule{0.350pt}{0.400pt}}
\put(1144,365.67){\rule{0.482pt}{0.400pt}}
\multiput(1144.00,366.17)(1.000,-1.000){2}{\rule{0.241pt}{0.400pt}}
\put(1146,364.17){\rule{0.700pt}{0.400pt}}
\multiput(1146.00,365.17)(1.547,-2.000){2}{\rule{0.350pt}{0.400pt}}
\put(1149,362.67){\rule{0.482pt}{0.400pt}}
\multiput(1149.00,363.17)(1.000,-1.000){2}{\rule{0.241pt}{0.400pt}}
\put(1151,361.17){\rule{0.700pt}{0.400pt}}
\multiput(1151.00,362.17)(1.547,-2.000){2}{\rule{0.350pt}{0.400pt}}
\put(1154,359.67){\rule{0.482pt}{0.400pt}}
\multiput(1154.00,360.17)(1.000,-1.000){2}{\rule{0.241pt}{0.400pt}}
\put(1156,358.67){\rule{0.482pt}{0.400pt}}
\multiput(1156.00,359.17)(1.000,-1.000){2}{\rule{0.241pt}{0.400pt}}
\put(1158,357.17){\rule{0.700pt}{0.400pt}}
\multiput(1158.00,358.17)(1.547,-2.000){2}{\rule{0.350pt}{0.400pt}}
\put(1161,355.67){\rule{0.482pt}{0.400pt}}
\multiput(1161.00,356.17)(1.000,-1.000){2}{\rule{0.241pt}{0.400pt}}
\put(1163,354.17){\rule{0.700pt}{0.400pt}}
\multiput(1163.00,355.17)(1.547,-2.000){2}{\rule{0.350pt}{0.400pt}}
\put(1166,352.67){\rule{0.482pt}{0.400pt}}
\multiput(1166.00,353.17)(1.000,-1.000){2}{\rule{0.241pt}{0.400pt}}
\put(1168,351.17){\rule{0.700pt}{0.400pt}}
\multiput(1168.00,352.17)(1.547,-2.000){2}{\rule{0.350pt}{0.400pt}}
\put(1171,349.67){\rule{0.482pt}{0.400pt}}
\multiput(1171.00,350.17)(1.000,-1.000){2}{\rule{0.241pt}{0.400pt}}
\put(1173,348.17){\rule{0.700pt}{0.400pt}}
\multiput(1173.00,349.17)(1.547,-2.000){2}{\rule{0.350pt}{0.400pt}}
\put(1176,346.67){\rule{0.482pt}{0.400pt}}
\multiput(1176.00,347.17)(1.000,-1.000){2}{\rule{0.241pt}{0.400pt}}
\put(1178,345.17){\rule{0.700pt}{0.400pt}}
\multiput(1178.00,346.17)(1.547,-2.000){2}{\rule{0.350pt}{0.400pt}}
\put(1181,343.67){\rule{0.482pt}{0.400pt}}
\multiput(1181.00,344.17)(1.000,-1.000){2}{\rule{0.241pt}{0.400pt}}
\put(1183,342.17){\rule{0.700pt}{0.400pt}}
\multiput(1183.00,343.17)(1.547,-2.000){2}{\rule{0.350pt}{0.400pt}}
\put(1186,340.67){\rule{0.482pt}{0.400pt}}
\multiput(1186.00,341.17)(1.000,-1.000){2}{\rule{0.241pt}{0.400pt}}
\put(1188,339.17){\rule{0.482pt}{0.400pt}}
\multiput(1188.00,340.17)(1.000,-2.000){2}{\rule{0.241pt}{0.400pt}}
\put(1190,337.67){\rule{0.723pt}{0.400pt}}
\multiput(1190.00,338.17)(1.500,-1.000){2}{\rule{0.361pt}{0.400pt}}
\put(1193,336.17){\rule{0.482pt}{0.400pt}}
\multiput(1193.00,337.17)(1.000,-2.000){2}{\rule{0.241pt}{0.400pt}}
\put(1195,334.67){\rule{0.723pt}{0.400pt}}
\multiput(1195.00,335.17)(1.500,-1.000){2}{\rule{0.361pt}{0.400pt}}
\put(1198,333.17){\rule{0.482pt}{0.400pt}}
\multiput(1198.00,334.17)(1.000,-2.000){2}{\rule{0.241pt}{0.400pt}}
\put(1200,331.17){\rule{0.700pt}{0.400pt}}
\multiput(1200.00,332.17)(1.547,-2.000){2}{\rule{0.350pt}{0.400pt}}
\put(1203,329.67){\rule{0.482pt}{0.400pt}}
\multiput(1203.00,330.17)(1.000,-1.000){2}{\rule{0.241pt}{0.400pt}}
\put(1205,328.17){\rule{0.700pt}{0.400pt}}
\multiput(1205.00,329.17)(1.547,-2.000){2}{\rule{0.350pt}{0.400pt}}
\put(1208,326.67){\rule{0.482pt}{0.400pt}}
\multiput(1208.00,327.17)(1.000,-1.000){2}{\rule{0.241pt}{0.400pt}}
\put(1210,325.17){\rule{0.700pt}{0.400pt}}
\multiput(1210.00,326.17)(1.547,-2.000){2}{\rule{0.350pt}{0.400pt}}
\put(1213,323.67){\rule{0.482pt}{0.400pt}}
\multiput(1213.00,324.17)(1.000,-1.000){2}{\rule{0.241pt}{0.400pt}}
\put(1215,322.17){\rule{0.700pt}{0.400pt}}
\multiput(1215.00,323.17)(1.547,-2.000){2}{\rule{0.350pt}{0.400pt}}
\put(1218,320.67){\rule{0.482pt}{0.400pt}}
\multiput(1218.00,321.17)(1.000,-1.000){2}{\rule{0.241pt}{0.400pt}}
\put(1220,319.17){\rule{0.482pt}{0.400pt}}
\multiput(1220.00,320.17)(1.000,-2.000){2}{\rule{0.241pt}{0.400pt}}
\put(1222,317.67){\rule{0.723pt}{0.400pt}}
\multiput(1222.00,318.17)(1.500,-1.000){2}{\rule{0.361pt}{0.400pt}}
\put(1225,316.17){\rule{0.482pt}{0.400pt}}
\multiput(1225.00,317.17)(1.000,-2.000){2}{\rule{0.241pt}{0.400pt}}
\put(1227,314.17){\rule{0.700pt}{0.400pt}}
\multiput(1227.00,315.17)(1.547,-2.000){2}{\rule{0.350pt}{0.400pt}}
\put(1230,312.67){\rule{0.482pt}{0.400pt}}
\multiput(1230.00,313.17)(1.000,-1.000){2}{\rule{0.241pt}{0.400pt}}
\put(1232,311.17){\rule{0.700pt}{0.400pt}}
\multiput(1232.00,312.17)(1.547,-2.000){2}{\rule{0.350pt}{0.400pt}}
\put(1235,309.67){\rule{0.482pt}{0.400pt}}
\multiput(1235.00,310.17)(1.000,-1.000){2}{\rule{0.241pt}{0.400pt}}
\put(1237,308.17){\rule{0.700pt}{0.400pt}}
\multiput(1237.00,309.17)(1.547,-2.000){2}{\rule{0.350pt}{0.400pt}}
\put(1240,306.67){\rule{0.482pt}{0.400pt}}
\multiput(1240.00,307.17)(1.000,-1.000){2}{\rule{0.241pt}{0.400pt}}
\put(1242,305.17){\rule{0.700pt}{0.400pt}}
\multiput(1242.00,306.17)(1.547,-2.000){2}{\rule{0.350pt}{0.400pt}}
\put(1245,303.17){\rule{0.482pt}{0.400pt}}
\multiput(1245.00,304.17)(1.000,-2.000){2}{\rule{0.241pt}{0.400pt}}
\put(1247,301.67){\rule{0.723pt}{0.400pt}}
\multiput(1247.00,302.17)(1.500,-1.000){2}{\rule{0.361pt}{0.400pt}}
\put(1250,300.17){\rule{0.482pt}{0.400pt}}
\multiput(1250.00,301.17)(1.000,-2.000){2}{\rule{0.241pt}{0.400pt}}
\put(1252,298.67){\rule{0.482pt}{0.400pt}}
\multiput(1252.00,299.17)(1.000,-1.000){2}{\rule{0.241pt}{0.400pt}}
\put(1254,297.17){\rule{0.700pt}{0.400pt}}
\multiput(1254.00,298.17)(1.547,-2.000){2}{\rule{0.350pt}{0.400pt}}
\put(1257,295.67){\rule{0.482pt}{0.400pt}}
\multiput(1257.00,296.17)(1.000,-1.000){2}{\rule{0.241pt}{0.400pt}}
\put(1259,294.17){\rule{0.700pt}{0.400pt}}
\multiput(1259.00,295.17)(1.547,-2.000){2}{\rule{0.350pt}{0.400pt}}
\put(1262,292.17){\rule{0.482pt}{0.400pt}}
\multiput(1262.00,293.17)(1.000,-2.000){2}{\rule{0.241pt}{0.400pt}}
\put(1264,290.67){\rule{0.723pt}{0.400pt}}
\multiput(1264.00,291.17)(1.500,-1.000){2}{\rule{0.361pt}{0.400pt}}
\put(1267,289.17){\rule{0.482pt}{0.400pt}}
\multiput(1267.00,290.17)(1.000,-2.000){2}{\rule{0.241pt}{0.400pt}}
\put(1269,287.67){\rule{0.723pt}{0.400pt}}
\multiput(1269.00,288.17)(1.500,-1.000){2}{\rule{0.361pt}{0.400pt}}
\put(1272,286.17){\rule{0.482pt}{0.400pt}}
\multiput(1272.00,287.17)(1.000,-2.000){2}{\rule{0.241pt}{0.400pt}}
\put(1274,284.17){\rule{0.700pt}{0.400pt}}
\multiput(1274.00,285.17)(1.547,-2.000){2}{\rule{0.350pt}{0.400pt}}
\put(1277,282.67){\rule{0.482pt}{0.400pt}}
\multiput(1277.00,283.17)(1.000,-1.000){2}{\rule{0.241pt}{0.400pt}}
\put(1279,281.17){\rule{0.700pt}{0.400pt}}
\multiput(1279.00,282.17)(1.547,-2.000){2}{\rule{0.350pt}{0.400pt}}
\put(1282,279.17){\rule{0.482pt}{0.400pt}}
\multiput(1282.00,280.17)(1.000,-2.000){2}{\rule{0.241pt}{0.400pt}}
\put(1284,277.67){\rule{0.482pt}{0.400pt}}
\multiput(1284.00,278.17)(1.000,-1.000){2}{\rule{0.241pt}{0.400pt}}
\put(1286,276.17){\rule{0.700pt}{0.400pt}}
\multiput(1286.00,277.17)(1.547,-2.000){2}{\rule{0.350pt}{0.400pt}}
\put(1289,274.67){\rule{0.482pt}{0.400pt}}
\multiput(1289.00,275.17)(1.000,-1.000){2}{\rule{0.241pt}{0.400pt}}
\put(1291,273.17){\rule{0.700pt}{0.400pt}}
\multiput(1291.00,274.17)(1.547,-2.000){2}{\rule{0.350pt}{0.400pt}}
\put(1294,271.17){\rule{0.482pt}{0.400pt}}
\multiput(1294.00,272.17)(1.000,-2.000){2}{\rule{0.241pt}{0.400pt}}
\put(1296,269.67){\rule{0.723pt}{0.400pt}}
\multiput(1296.00,270.17)(1.500,-1.000){2}{\rule{0.361pt}{0.400pt}}
\put(1299,268.17){\rule{0.482pt}{0.400pt}}
\multiput(1299.00,269.17)(1.000,-2.000){2}{\rule{0.241pt}{0.400pt}}
\put(1301,266.17){\rule{0.700pt}{0.400pt}}
\multiput(1301.00,267.17)(1.547,-2.000){2}{\rule{0.350pt}{0.400pt}}
\put(1304,264.67){\rule{0.482pt}{0.400pt}}
\multiput(1304.00,265.17)(1.000,-1.000){2}{\rule{0.241pt}{0.400pt}}
\put(1306,263.17){\rule{0.700pt}{0.400pt}}
\multiput(1306.00,264.17)(1.547,-2.000){2}{\rule{0.350pt}{0.400pt}}
\put(1309,261.17){\rule{0.482pt}{0.400pt}}
\multiput(1309.00,262.17)(1.000,-2.000){2}{\rule{0.241pt}{0.400pt}}
\put(1311,259.67){\rule{0.482pt}{0.400pt}}
\multiput(1311.00,260.17)(1.000,-1.000){2}{\rule{0.241pt}{0.400pt}}
\put(1313,258.17){\rule{0.700pt}{0.400pt}}
\multiput(1313.00,259.17)(1.547,-2.000){2}{\rule{0.350pt}{0.400pt}}
\put(1316,256.17){\rule{0.482pt}{0.400pt}}
\multiput(1316.00,257.17)(1.000,-2.000){2}{\rule{0.241pt}{0.400pt}}
\put(1318,254.67){\rule{0.723pt}{0.400pt}}
\multiput(1318.00,255.17)(1.500,-1.000){2}{\rule{0.361pt}{0.400pt}}
\put(1321,253.17){\rule{0.482pt}{0.400pt}}
\multiput(1321.00,254.17)(1.000,-2.000){2}{\rule{0.241pt}{0.400pt}}
\put(1323,251.17){\rule{0.700pt}{0.400pt}}
\multiput(1323.00,252.17)(1.547,-2.000){2}{\rule{0.350pt}{0.400pt}}
\put(1326,249.67){\rule{0.482pt}{0.400pt}}
\multiput(1326.00,250.17)(1.000,-1.000){2}{\rule{0.241pt}{0.400pt}}
\put(1328,248.17){\rule{0.700pt}{0.400pt}}
\multiput(1328.00,249.17)(1.547,-2.000){2}{\rule{0.350pt}{0.400pt}}
\put(1331,246.17){\rule{0.482pt}{0.400pt}}
\multiput(1331.00,247.17)(1.000,-2.000){2}{\rule{0.241pt}{0.400pt}}
\put(1333,244.67){\rule{0.723pt}{0.400pt}}
\multiput(1333.00,245.17)(1.500,-1.000){2}{\rule{0.361pt}{0.400pt}}
\put(1336,243.17){\rule{0.482pt}{0.400pt}}
\multiput(1336.00,244.17)(1.000,-2.000){2}{\rule{0.241pt}{0.400pt}}
\put(1338,241.17){\rule{0.700pt}{0.400pt}}
\multiput(1338.00,242.17)(1.547,-2.000){2}{\rule{0.350pt}{0.400pt}}
\put(1341,239.67){\rule{0.482pt}{0.400pt}}
\multiput(1341.00,240.17)(1.000,-1.000){2}{\rule{0.241pt}{0.400pt}}
\put(1343,238.17){\rule{0.482pt}{0.400pt}}
\multiput(1343.00,239.17)(1.000,-2.000){2}{\rule{0.241pt}{0.400pt}}
\put(1345,236.17){\rule{0.700pt}{0.400pt}}
\multiput(1345.00,237.17)(1.547,-2.000){2}{\rule{0.350pt}{0.400pt}}
\put(1348,234.67){\rule{0.482pt}{0.400pt}}
\multiput(1348.00,235.17)(1.000,-1.000){2}{\rule{0.241pt}{0.400pt}}
\put(1350,233.17){\rule{0.700pt}{0.400pt}}
\multiput(1350.00,234.17)(1.547,-2.000){2}{\rule{0.350pt}{0.400pt}}
\put(1353,231.17){\rule{0.482pt}{0.400pt}}
\multiput(1353.00,232.17)(1.000,-2.000){2}{\rule{0.241pt}{0.400pt}}
\put(1355,229.67){\rule{0.723pt}{0.400pt}}
\multiput(1355.00,230.17)(1.500,-1.000){2}{\rule{0.361pt}{0.400pt}}
\put(1358,228.17){\rule{0.482pt}{0.400pt}}
\multiput(1358.00,229.17)(1.000,-2.000){2}{\rule{0.241pt}{0.400pt}}
\put(1360,226.17){\rule{0.700pt}{0.400pt}}
\multiput(1360.00,227.17)(1.547,-2.000){2}{\rule{0.350pt}{0.400pt}}
\put(1363,224.17){\rule{0.482pt}{0.400pt}}
\multiput(1363.00,225.17)(1.000,-2.000){2}{\rule{0.241pt}{0.400pt}}
\put(1365,222.67){\rule{0.723pt}{0.400pt}}
\multiput(1365.00,223.17)(1.500,-1.000){2}{\rule{0.361pt}{0.400pt}}
\put(1368,221.17){\rule{0.482pt}{0.400pt}}
\multiput(1368.00,222.17)(1.000,-2.000){2}{\rule{0.241pt}{0.400pt}}
\put(1370,219.17){\rule{0.700pt}{0.400pt}}
\multiput(1370.00,220.17)(1.547,-2.000){2}{\rule{0.350pt}{0.400pt}}
\put(1373,217.67){\rule{0.482pt}{0.400pt}}
\multiput(1373.00,218.17)(1.000,-1.000){2}{\rule{0.241pt}{0.400pt}}
\put(1375,216.17){\rule{0.482pt}{0.400pt}}
\multiput(1375.00,217.17)(1.000,-2.000){2}{\rule{0.241pt}{0.400pt}}
\put(1377,214.17){\rule{0.700pt}{0.400pt}}
\multiput(1377.00,215.17)(1.547,-2.000){2}{\rule{0.350pt}{0.400pt}}
\put(1380,212.17){\rule{0.482pt}{0.400pt}}
\multiput(1380.00,213.17)(1.000,-2.000){2}{\rule{0.241pt}{0.400pt}}
\put(1382,210.67){\rule{0.723pt}{0.400pt}}
\multiput(1382.00,211.17)(1.500,-1.000){2}{\rule{0.361pt}{0.400pt}}
\put(1385,209.17){\rule{0.482pt}{0.400pt}}
\multiput(1385.00,210.17)(1.000,-2.000){2}{\rule{0.241pt}{0.400pt}}
\put(1387,207.17){\rule{0.700pt}{0.400pt}}
\multiput(1387.00,208.17)(1.547,-2.000){2}{\rule{0.350pt}{0.400pt}}
\put(1390,205.17){\rule{0.482pt}{0.400pt}}
\multiput(1390.00,206.17)(1.000,-2.000){2}{\rule{0.241pt}{0.400pt}}
\put(1392,203.67){\rule{0.723pt}{0.400pt}}
\multiput(1392.00,204.17)(1.500,-1.000){2}{\rule{0.361pt}{0.400pt}}
\put(1395,202.17){\rule{0.482pt}{0.400pt}}
\multiput(1395.00,203.17)(1.000,-2.000){2}{\rule{0.241pt}{0.400pt}}
\put(1397,200.17){\rule{0.700pt}{0.400pt}}
\multiput(1397.00,201.17)(1.547,-2.000){2}{\rule{0.350pt}{0.400pt}}
\put(1400,198.17){\rule{0.482pt}{0.400pt}}
\multiput(1400.00,199.17)(1.000,-2.000){2}{\rule{0.241pt}{0.400pt}}
\put(1402,196.67){\rule{0.723pt}{0.400pt}}
\multiput(1402.00,197.17)(1.500,-1.000){2}{\rule{0.361pt}{0.400pt}}
\put(1405,195.17){\rule{0.482pt}{0.400pt}}
\multiput(1405.00,196.17)(1.000,-2.000){2}{\rule{0.241pt}{0.400pt}}
\put(1407,193.17){\rule{0.482pt}{0.400pt}}
\multiput(1407.00,194.17)(1.000,-2.000){2}{\rule{0.241pt}{0.400pt}}
\put(1409,191.17){\rule{0.700pt}{0.400pt}}
\multiput(1409.00,192.17)(1.547,-2.000){2}{\rule{0.350pt}{0.400pt}}
\put(1412,189.67){\rule{0.482pt}{0.400pt}}
\multiput(1412.00,190.17)(1.000,-1.000){2}{\rule{0.241pt}{0.400pt}}
\put(1414,188.17){\rule{0.700pt}{0.400pt}}
\multiput(1414.00,189.17)(1.547,-2.000){2}{\rule{0.350pt}{0.400pt}}
\put(1417,186.17){\rule{0.482pt}{0.400pt}}
\multiput(1417.00,187.17)(1.000,-2.000){2}{\rule{0.241pt}{0.400pt}}
\put(1419,184.17){\rule{0.700pt}{0.400pt}}
\multiput(1419.00,185.17)(1.547,-2.000){2}{\rule{0.350pt}{0.400pt}}
\put(1422,182.67){\rule{0.482pt}{0.400pt}}
\multiput(1422.00,183.17)(1.000,-1.000){2}{\rule{0.241pt}{0.400pt}}
\put(1424,181.17){\rule{0.700pt}{0.400pt}}
\multiput(1424.00,182.17)(1.547,-2.000){2}{\rule{0.350pt}{0.400pt}}
\put(1427,179.17){\rule{0.482pt}{0.400pt}}
\multiput(1427.00,180.17)(1.000,-2.000){2}{\rule{0.241pt}{0.400pt}}
\put(1429,177.17){\rule{0.700pt}{0.400pt}}
\multiput(1429.00,178.17)(1.547,-2.000){2}{\rule{0.350pt}{0.400pt}}
\put(1432,175.17){\rule{0.482pt}{0.400pt}}
\multiput(1432.00,176.17)(1.000,-2.000){2}{\rule{0.241pt}{0.400pt}}
\put(1434,173.67){\rule{0.723pt}{0.400pt}}
\multiput(1434.00,174.17)(1.500,-1.000){2}{\rule{0.361pt}{0.400pt}}
\put(1437,172.17){\rule{0.482pt}{0.400pt}}
\multiput(1437.00,173.17)(1.000,-2.000){2}{\rule{0.241pt}{0.400pt}}
\put(570.0,642.0){\rule[-0.200pt]{0.723pt}{0.400pt}}
\put(1279,778){\makebox(0,0)[r]{$\beta = 2/3$}}
\multiput(1299,778)(20.756,0.000){5}{\usebox{\plotpoint}}
\put(1399,778){\usebox{\plotpoint}}
\put(211,848){\usebox{\plotpoint}}
\put(211.00,848.00){\usebox{\plotpoint}}
\put(230.58,842.14){\usebox{\plotpoint}}
\put(250.04,835.99){\usebox{\plotpoint}}
\put(269.58,830.21){\usebox{\plotpoint}}
\put(289.21,824.26){\usebox{\plotpoint}}
\put(308.76,818.12){\usebox{\plotpoint}}
\put(328.30,812.35){\usebox{\plotpoint}}
\put(347.70,806.00){\usebox{\plotpoint}}
\put(367.19,800.00){\usebox{\plotpoint}}
\put(386.67,793.67){\usebox{\plotpoint}}
\put(406.10,787.63){\usebox{\plotpoint}}
\put(425.76,781.75){\usebox{\plotpoint}}
\put(445.17,774.91){\usebox{\plotpoint}}
\put(464.65,768.78){\usebox{\plotpoint}}
\put(484.08,761.97){\usebox{\plotpoint}}
\put(503.73,756.09){\usebox{\plotpoint}}
\put(523.16,749.28){\usebox{\plotpoint}}
\put(542.67,743.00){\usebox{\plotpoint}}
\put(562.12,736.29){\usebox{\plotpoint}}
\put(581.55,729.48){\usebox{\plotpoint}}
\put(600.92,722.54){\usebox{\plotpoint}}
\put(620.38,715.81){\usebox{\plotpoint}}
\put(639.64,708.79){\usebox{\plotpoint}}
\put(659.00,702.00){\usebox{\plotpoint}}
\put(678.43,695.19){\usebox{\plotpoint}}
\put(697.69,688.16){\usebox{\plotpoint}}
\put(717.10,681.30){\usebox{\plotpoint}}
\put(736.30,673.57){\usebox{\plotpoint}}
\put(755.69,666.65){\usebox{\plotpoint}}
\put(774.94,659.02){\usebox{\plotpoint}}
\put(794.18,651.94){\usebox{\plotpoint}}
\put(813.38,644.21){\usebox{\plotpoint}}
\put(832.59,636.47){\usebox{\plotpoint}}
\put(851.75,628.63){\usebox{\plotpoint}}
\put(871.00,621.00){\usebox{\plotpoint}}
\put(890.08,612.97){\usebox{\plotpoint}}
\put(909.28,605.24){\usebox{\plotpoint}}
\put(928.49,597.50){\usebox{\plotpoint}}
\put(947.66,589.67){\usebox{\plotpoint}}
\put(966.90,582.03){\usebox{\plotpoint}}
\put(985.45,573.27){\usebox{\plotpoint}}
\put(1004.43,565.05){\usebox{\plotpoint}}
\put(1023.41,556.86){\usebox{\plotpoint}}
\put(1042.20,548.27){\usebox{\plotpoint}}
\put(1060.92,539.54){\usebox{\plotpoint}}
\put(1079.79,531.14){\usebox{\plotpoint}}
\put(1098.46,522.27){\usebox{\plotpoint}}
\put(1117.18,513.91){\usebox{\plotpoint}}
\put(1135.73,505.13){\usebox{\plotpoint}}
\put(1154.45,496.77){\usebox{\plotpoint}}
\put(1173.12,487.92){\usebox{\plotpoint}}
\put(1191.37,478.54){\usebox{\plotpoint}}
\put(1210.01,470.00){\usebox{\plotpoint}}
\put(1228.37,460.54){\usebox{\plotpoint}}
\put(1246.70,451.30){\usebox{\plotpoint}}
\put(1265.21,442.19){\usebox{\plotpoint}}
\put(1283.76,433.12){\usebox{\plotpoint}}
\put(1301.82,423.73){\usebox{\plotpoint}}
\put(1320.05,414.32){\usebox{\plotpoint}}
\put(1338.11,404.92){\usebox{\plotpoint}}
\put(1356.19,395.21){\usebox{\plotpoint}}
\put(1374.37,385.63){\usebox{\plotpoint}}
\put(1392.38,375.87){\usebox{\plotpoint}}
\put(1410.46,366.51){\usebox{\plotpoint}}
\put(1428.34,356.66){\usebox{\plotpoint}}
\put(1439,351){\usebox{\plotpoint}}
\put(211.0,131.0){\rule[-0.200pt]{0.400pt}{175.375pt}}
\put(211.0,131.0){\rule[-0.200pt]{295.825pt}{0.400pt}}
\put(1439.0,131.0){\rule[-0.200pt]{0.400pt}{175.375pt}}
\put(211.0,859.0){\rule[-0.200pt]{295.825pt}{0.400pt}}
\end{picture}

\end{figure}

\end{document}