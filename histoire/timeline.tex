% Chronologie de la cosmologie moderne (post relativité)
% Evenements seulement, succinct
% (découvertes majeures)
% https://cosmology.carnegiescience.edu/timeline

% TODO:
% - Steady Universe
% - MQ age univ <~= 1/H0

\documentclass[10pt]{article}

\usepackage[paperwidth=297mm,%
    paperheight=480mm,%
    margin=5mm,%
    vscale=1,%
    hscale=1]{geometry}
%\usepackage[a3paper,margin=3mm]{geometry}

\usepackage[utf8]{inputenc}
\usepackage[T1]{fontenc}

\usepackage{tikz}

\usetikzlibrary{arrows, calc, decorations.markings, positioning}

\pagestyle{empty}

\makeatletter
\newenvironment{timeline}[6]{%
    % #1 is startyear
    % #2 is tlendyear
    % #3 is yearcolumnwidth
    % #4 is rulecolumnwidth
    % #5 is entrycolumnwidth
    % #6 is timelineheight

    \newcommand{\startyear}{#1}
    \newcommand{\tlendyear}{#2}

    \newcommand{\yearcolumnwidth}{#3}
    \newcommand{\rulecolumnwidth}{#4}
    \newcommand{\entrycolumnwidth}{#5}
    \newcommand{\timelineheight}{#6}

    \newcommand{\templength}{}

    \newcommand{\entrycounter}{0}

    % http://tex.stackexchange.com/questions/85528/checking-whether-or-not-a-node-has-been-previously-defined
    % http://tex.stackexchange.com/questions/37709/how-can-i-know-if-a-node-is-already-defined
    \long\def\ifnodedefined##1##2##3{%
        \@ifundefined{pgf@sh@ns@##1}{##3}{##2}%
    }

    \newcommand{\ifnodeundefined}[2]{%
        \ifnodedefined{##1}{}{##2}
    }

    \newcommand{\drawtimeline}{%
        \draw[timelinerule] (\yearcolumnwidth+5pt, 0pt) -- (\yearcolumnwidth+5pt, -\timelineheight);
        \draw (\yearcolumnwidth+0pt, -10pt) -- (\yearcolumnwidth+10pt, -10pt);
        \draw (\yearcolumnwidth+0pt, -\timelineheight+15pt) -- (\yearcolumnwidth+10pt, -\timelineheight+15pt);

        \pgfmathsetlengthmacro{\templength}{neg(add(multiply(subtract(\startyear, \startyear), divide(subtract(\timelineheight, 25), subtract(\tlendyear, \startyear))), 10))}
        \node[year] (year-\startyear) at (\yearcolumnwidth, \templength) {\startyear};

        \pgfmathsetlengthmacro{\templength}{neg(add(multiply(subtract(\tlendyear, \startyear), divide(subtract(\timelineheight, 25), subtract(\tlendyear, \startyear))), 10))}
        \node[year] (year-\tlendyear) at (\yearcolumnwidth, \templength) {\tlendyear};
    }

    \newcommand{\entry}[2]{%
        % #1 is the year
        % #2 is the entry text

        \pgfmathtruncatemacro{\lastentrycount}{\entrycounter}
        \pgfmathtruncatemacro{\entrycounter}{\entrycounter + 1}

        \ifdim \lastentrycount pt > 0 pt%
            \node[entry] (entry-\entrycounter) [below of=entry-\lastentrycount] {##2};
        \else%
            \pgfmathsetlengthmacro{\templength}{neg(add(multiply(subtract(\startyear, \startyear), divide(subtract(\timelineheight, 25), subtract(\tlendyear, \startyear))), 10))}
            \node[entry] (entry-\entrycounter) at (\yearcolumnwidth+\rulecolumnwidth+10pt, \templength) {##2};
        \fi

        \ifnodeundefined{year-##1}{%
            \pgfmathsetlengthmacro{\templength}{neg(add(multiply(subtract(##1, \startyear), divide(subtract(\timelineheight, 25), subtract(\tlendyear, \startyear))), 10))}
            \draw (\yearcolumnwidth+2.5pt, \templength) -- (\yearcolumnwidth+7.5pt, \templength);
            \node[year] (year-##1) at (\yearcolumnwidth, \templength) {##1};
        }

        \draw ($(year-##1.east)+(2.5pt, 0pt)$) -- ($(year-##1.east)+(7.5pt, 0pt)$) -- ($(entry-\entrycounter.west)-(5pt,0)$) -- (entry-\entrycounter.west);
    }

    \newcommand{\plainentry}[2]{% plainentry won't print date in the timeline
        % #1 is the year
        % #2 is the entry text

        \pgfmathtruncatemacro{\lastentrycount}{\entrycounter}
        \pgfmathtruncatemacro{\entrycounter}{\entrycounter + 1}

        \ifdim \lastentrycount pt > 0 pt%
            \node[entry] (entry-\entrycounter) [below of=entry-\lastentrycount] {##2};
        \else%
            \pgfmathsetlengthmacro{\templength}{neg(add(multiply(subtract(\startyear, \startyear), divide(subtract(\timelineheight, 25), subtract(\tlendyear, \startyear))), 10))}
            \node[entry] (entry-\entrycounter) at (\yearcolumnwidth+\rulecolumnwidth+10pt, \templength) {##2};
        \fi

        \ifnodeundefined{invisible-year-##1}{%
            \pgfmathsetlengthmacro{\templength}{neg(add(multiply(subtract(##1, \startyear), divide(subtract(\timelineheight, 25), subtract(\tlendyear, \startyear))), 10))}
            \draw (\yearcolumnwidth+2.5pt, \templength) -- (\yearcolumnwidth+7.5pt, \templength);
            \node[year] (invisible-year-##1) at (\yearcolumnwidth, \templength) {};
        }

        \draw ($(invisible-year-##1.east)+(2.5pt, 0pt)$) -- ($(invisible-year-##1.east)+(7.5pt, 0pt)$) -- ($(entry-\entrycounter.west)-(5pt,0)$) -- (entry-\entrycounter.west);
    }

    \begin{tikzpicture}
        \tikzstyle{entry} = [%
            align=left,%
            text width=\entrycolumnwidth,%
            node distance=10mm,%
            anchor=west]
        \tikzstyle{year} = [anchor=east]
        \tikzstyle{timelinerule} = [%
            draw,%
            decoration={markings, mark=at position 1 with {\arrow[scale=1.5]{latex'}}},%
            postaction={decorate},%
            shorten >=0.4pt]

        \drawtimeline
}
{
    \end{tikzpicture}
    \let\startyear\@undefined
    \let\tlendyear\@undefined
    \let\yearcolumnwidth\@undefined
    \let\rulecolumnwidth\@undefined
    \let\entrycolumnwidth\@undefined
    \let\timelineheight\@undefined
    \let\entrycounter\@undefined
    \let\ifnodedefined\@undefined
    \let\ifnodeundefined\@undefined
    \let\drawtimeline\@undefined
    \let\entry\@undefined
}
\makeatother


\begin{document}

\bigskip

\smallskip

\begin{timeline}{1905}{2015}{1.5cm}{3cm}{35cm}{0.45\textheight}
\entry{1905}{Albert Einstein publie sa théorie de la Relativité Restreinte qui décrit l'électrodynamique classique }
\entry{1912}{Vesto M. Slipher découvre grâce au \textit{red-shift} l'éloignement des galaxies spirales de la Terre}
\entry{1915}{Albert Einstein et David Hilbert finalisent la théorie de la Relativité Générale, une théorie relativiste de la gravitation}
\entry{1917}{A. Einstein introduit la constante cosmologique dans l'équation qui porte son nom afin de la rendre compatible avec un Univers statique}
\entry{1922}{Alexandre Friedmann publie une théorie décrivant un Univers en expansion de courure positive basé sur la Relativité Générale}
\entry{1927}{Georges Lemaître publie un article dans lequel il décrit un Univers en expansion homogène isotrope de masse constante et prédit la future "Loi de Hubble" en expliquant la fuite des galaxies par cette expansion }
\entry{1929}{Découverte de l'expansion de l'Univers par Hubble et preuve expérimentale de la loi portant son nom}
\entry{1931}{G. Lemaître suggère que l'expansion de l'Univers l'a amené d'une taille arbitrairement petite à sa taille actuelle en un temps fini, qu'il explique par la désintégration d'un état dense appelé "atome primitif"}
\entry{1932}{Yann Oort propose l'existence de matière noire pour expliquer la vitesse des étoiles de la voie lactée}
\entry{1948}{F. Hoyle propose un modèle stationnaire de l'Univers, en accord avec le principe cosmologique parfait, dans lequel une création continue de matière compense la diminution de densité due à l'expansion, en contradiction avec le modèle de Lemaître que Hoyle surnomme théorie du "Big Bang"}
\entry{1948}{Ralph Alpher et Robert Herman suggèrent un rayonnement fossile émis au découplage de la lumière et de la matière dans le cadre du modèle du Big Bang}
\entry{1952}{Walter Baade découvre un nouveau genre d'étoile Céphéide variable, impliquant une nouvelle valeur de la constante de Hubble. D'autres corrections apportées à la mesure de cette constante permettent de rendre l'estimation de l'âge de l'Univers davantage compatible avec celui des diffèrents objets célestes. }
\entry{1963}{Maarten Schmidt découvre un nouveau type d'objet astronomique plus tard appelé "Quasar". Les observations montrent qu'on en trouve seulement à une distance importante, ce sont donc des objets anciens, en contradiction avec le principe cosmologique parfait, ce qui porte un coup au modèle stationnaire de l'Univers.  }
\entry{1964}{Découverte accidentelle du fond diffus cosmologique par Arno Penzias et Robert Wilson. }
\entry{1966}{Rainer K. Sachs et Arthur M. Wolfe décrivent l'effet Sachs-Wolfe, qui relie les fluctuations directionnelles de densité de matière aux fluctuations de température du rayonnement fossile }
\entry{1980}{Les travaux menés par Kent Ford et Vera Rubin montrent que la plupart des galaxies qu'ils ont pu étudier sont principellement constituées de matière noire}
% DM Ford-Rubin : 1960-1980
\entry{1990}{Des expériences menées grâce au satellite KOBE permettent une mesure précise du spectre du fond diffus cosmologique}
\entry{1998}{Découverte de l'accélération de l'expansion de l'Univers}
\entry{1998}{Introduction de la notion d'"Énergie sombre" par Dragan Huterer et Michael S. Turner pour expliquer l'accélération de l'expansion. Ceci comprend entre autres hypothèses la réintroduction de la constante cosmologique. }
\entry{2015}{Les résultats de l'expérience Planck ... }
\end{timeline}

\end{document}