% !TEX TS-program = pdflatex
% !TEX encoding = UTF-8 Unicode

% This is a simple template for a LaTeX document using the "article" class.
% See "book", "report", "letter" for other types of document.

\documentclass[11pt]{article} % use larger type; default would be 10pt

\usepackage[utf8]{inputenc} % set input encoding (not needed with XeLaTeX)

%%% Examples of Article customizations
% These packages are optional, depending whether you want the features they provide.
% See the LaTeX Companion or other references for full information.

%%% PAGE DIMENSIONS
\usepackage{geometry} % to change the page dimensions
\geometry{a4paper} % or letterpaper (US) or a5paper or....
% \geometry{margin=2in} % for example, change the margins to 2 inches all round
% \geometry{landscape} % set up the page for landscape
%   read geometry.pdf for detailed page layout information

\usepackage{amsmath}
\usepackage{amsfonts}
\usepackage{amssymb}
\usepackage{url}
\usepackage{eqnarray}
\usepackage{tikz}
\usepackage{esvect}
\usepackage{graphicx}
\usepackage{color}
\usepackage[french]{babel}

\usepackage{graphicx} % support the \includegraphics command and options
\usepackage[outdir=./]{epstopdf}
\usepackage{psfrag}
\usepackage{subcaption}

\usepackage{feynmp-auto}

\makeatletter
\def\endfmffile{%
  \fmfcmd{\p@rcent\space the end.^^J%
          end.^^J%
          endinput;}%
  \if@fmfio
    \immediate\closeout\@outfmf
  \fi
  \IfFileExists{\thefmffile.mp}{\immediate\write18{mpost \thefmffile}}{}
  \let\thefmffile\relax
}
\makeatother

% Déclarations



% Commandes

% d droit
\newcommand{\dd}{ \mathop{}\mathopen{}\mathrm{d}}

% opérateurs vectoriels
\newcommand{\grad}{\vec{\nabla}}
%\newcommand{\div}{\nabla \cdot}
\newcommand{\rot}{\vec{\nabla} \times }

% bra ket
\newcommand{\ket}[1]{|{#1}\rangle}
\newcommand{\bra}[1]{\langle{#1}|}

\newcommand{\sket}[1]{{#1}\rangle}
\newcommand{\sbra}[1]{\langle{#1}}

% Résidus
\DeclareMathOperator*{\res}{Res}

% displaystyle
\newcommand{\dint}{\displaystyle \int}
\newcommand{\dsum}{\displaystyle \sum}

% \usepackage[parfill]{parskip} % Activate to begin paragraphs with an empty line rather than an indent

%%% PACKAGES
\usepackage{booktabs} % for much better looking tables
\usepackage{array} % for better arrays (eg matrices) in maths
\usepackage{paralist} % very flexible & customisable lists (eg. enumerate/itemize, etc.)
\usepackage{verbatim} % adds environment for commenting out blocks of text & for better verbatim
\usepackage{subfig} % make it possible to include more than one captioned figure/table in a single float
% These packages are all incorporated in the memoir class to one degree or another...

%%% HEADERS & FOOTERS
\usepackage{fancyhdr} % This should be set AFTER setting up the page geometry
\pagestyle{fancy} % options: empty , plain , fancy
\renewcommand{\headrulewidth}{0pt} % customise the layout...
\lhead{}\chead{}\rhead{}
\lfoot{}\cfoot{\thepage}\rfoot{}

%%% SECTION TITLE APPEARANCE
\usepackage{sectsty}
\allsectionsfont{\sffamily\mdseries\upshape} % (See the fntguide.pdf for font help)
% (This matches ConTeXt defaults)

%%% ToC (table of contents) APPEARANCE
\usepackage[nottoc,notlof,notlot]{tocbibind} % Put the bibliography in the ToC
\usepackage[titles,subfigure]{tocloft} % Alter the style of the Table of Contents
\renewcommand{\cftsecfont}{\rmfamily\mdseries\upshape}
\renewcommand{\cftsecpagefont}{\rmfamily\mdseries\upshape} % No bold!

%%% END Article customizations

%%% The "real" document content comes below...

\title{Notes}
\author{}
\date{} % Activate to display a given date or no date (if empty),
         % otherwise the current date is printed 

\begin{document}
\maketitle

\section{Temps propagation lumière dans un espace en expansion}

A envoie un signal à B situé à une distance $d$ ($x(A) = 0$ et $x(B) = d$) dans un espace en expansion selon $t \mapsto L(t)$. 

Pour la lumière

\begin{equation}
c dt = \dfrac{L(t)}{L(0)} dx
\end{equation}

Donc \begin{equation}
d = cL_0 \dint_0^T \dfrac{dt}{L(t)}
\end{equation}


Exemple 1 : $L(t) = L_0 e^{\alpha t}$ :

 \begin{equation}
d = c \dint_0^T e^{-\alpha t} dt = \dfrac{c}{\alpha} \left ( 1 - e^{-\alpha T} \right )
\end{equation}

Donc $T = \dfrac{1}{\alpha} \ln { \dfrac{1}{1-\alpha d/c}}$ si $d \leq c/\alpha$.
De plus $T = d/c + O(\alpha d/c)$ pour $\alpha \to 0$

Exemple 2 : $L(t) = L_0 \left ( \dfrac{t}{t_0} \right )^\beta$ :

 \begin{equation}
d = c t_0 ^\beta \dint_0^T t^{-\beta} dt = \dfrac{ct_0^\beta} {1-\beta}  T^{1-\beta}  \textrm{ si } \beta < 1
\end{equation}

Donc $T = \left [ (1-\beta) \dfrac{d}{ct_0^\beta} \right ] ^ { \left [1/(1-\beta)\right ] }$

\section{Horizons}

\textbf{Horizon des évènements} : distance maximale $R_{ev}$ de laquelle peut provenir un signal lumineux en un temps fini.

Exemple 1 : $L(t) = L_0 e^{\alpha t} \Rightarrow R_{ev} = c/\alpha$

Exemple 2 : $L_0 \left ( \dfrac{t}{t_0} \right )^\beta, \beta < 1 \Rightarrow R_{ev} = +\infty$

\textbf{Horizon des particules} : distance physique maximale à un instant $t_1$ de laquelle peut provenir un évènement issu de $t_0$.

\begin{equation}
R_{part} = L(t_1) c \dint_{t_0}^{t_1} \dfrac{dt}{L(t)}
\end{equation}

\section{Relation luminosité-z}

On définit $z$ tq $1+z(t) = \dfrac{L(t)}{L_0}$.

Une source émet $N$ photons par seconde en $x = 0$. Au temps $T$ on capte ce signal en un point originellement ($t = 0$) distant de $d$ de la source sur une "surface" $dl$ du front d'onde l(T). On mesure $dN'(T)$ photons par seconde.

On doit avoir :
$dN'(T) = \dfrac{dl}{l(T)} \dfrac{f(T)}{f_0} N$


La longueur du front d'onde au temps $t$ est donnée par $l(t)$. Dans un espace plat + 2D :

\begin{equation}
l(T) = 2\pi d \dfrac{L(T)}{L_0} = 2\pi d (1+z(T)) = 2\pi (1+z(T)) \dint_0^T \dfrac{dt}{1+z(t)}
\end{equation}

Dans un espace sphérique 2D :
\begin{equation}
l(T) = 2\pi L(T) \sin \dfrac{d}{L_0} = 2\pi L_0 (1+z(T)) \sin \dfrac{d}{L_0}  = 2\pi L_0 (1+z(T)) \sin \dfrac{c}{L_0} \dint_0^T \dfrac{dt}{1+z(t)}
\end{equation}

L'énergie des photons décroit en $1+z$ (red shift) 

Espace plat 2D : 
\begin{equation}
\bar{L} (z) = \dfrac{\bar{L}}{2\pi d(1+z)^2}
\end{equation}

Espace sphérique 2D : 
\begin{equation}
\bar{L} (z) = \dfrac{\bar{L}}{2\pi L_0(1+z)^2 \sin\frac{d}{L_0}}
\end{equation}


\end{document}